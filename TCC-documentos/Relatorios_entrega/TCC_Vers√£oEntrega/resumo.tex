%% RESUMOS

%% Resumo em Portugu^es. OBRIGATORIO.
\setlength{\absparsep}{18pt} 
\begin{resumo}

\noindent 
A caracterização das superfícies de pavimentos mostra-se de extrema importância para garantir a segurança dos usuários e melhor funcionamento dos veículos nas rodovias. O desempenho dos pavimentos é avaliado e classificado a partir de vários parâmetros, sendo a rugosidade um dos principais, pois influencia diretamente a qualidade do contato entre pneu e asfalto. Atualmente, tal caracterização é feita por meio de ensaios indiretos, como o Ensaio de Mancha de Areia, que apresentam um alto nível de imprecisão; ou por meio de equipamentos de medição denominados perfilômetros, os quais apresentam um alto custo e cuja utilização é bastante limitada. Dessa forma, notou-se a necessidade de desenvolver métodos alternativos que sejam eficientes. O presente trabalho apresenta um método desenvolvido com base na tecnologia LiDAR (\emph{Light Detection and Ranging}), a qual utiliza-se de equipamentos de escaneamento tridimensional a laser para obter um modelo de um objeto. O método consiste em escanear a superfície de um corpo de prova extraído do pavimento e seccioná-lo por sucessivos planos transversais, obtendo-se perfis representativos da superfície. Posteriormente, o resultado é analisado por um algoritmo que realiza o cálculo da altura média dos perfis, resultando em um parâmetro que pode ser associado à medida de textura superficial obtida em campo. O emprego desta metodologia mostra-se promissor, as alturas de areia encontradas para os pavimentos de concreto e de asfalto foram respectivamente 1,000mm e 1,035mm, sendo condizentes com a análise visual; além disso o coeficiente de determinação $(R^2)$, que relaciona os valores experimentais e computacionais foi de 0,9815, indicando uma alta precisão.

\vspace{1cm}

\noindent {{\bfseries Palavras-chave:} pavimento asfáltico; rugosidade; textura superficial; irregularidade; escaneamento tridimensional; mancha de areia;}

\end{resumo}