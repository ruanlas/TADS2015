%% Resumo em Ingle^s
\begin{resumo}[ABSTRACT]
 \begin{otherlanguage*}{english}
   This project aims at studying embedded technologies found in cars in order to understand the communication patterns among the systems adopted by the automotive industry. Also, intends to explore means by which computer science can contribute to make clear the information involved and simplify or automate some analyses. In order to apply the concepts found, it was proposed the implementation of a system upon a low cost hardware platform (using a Raspberry Pi board) able to read information originated from cars’ sensors and to communicate with cloud computing services. This systems development was divided in three distinct fronts. The first one is related to the software in charge of obtain the vehicle's data; the second is responsible for prepare the Raspberry Pi's environment to execute, including the device configuration and software installation. The last one deals with the integration between de embedded software and the cloud computing platform. The whole development was designed under the object oriented paradigm, using Java, as well as architectural patterns Model-View-Controller (MVC) and Data Access Object (DAO). The cloud computing services adopted have used MongoDB as database management system and the Node.js framework for web service creation. The system allows generating data to be stored in the cloud and which can be made available to other applications involving diverse areas as, for example, Internet of Things and Big Data. The final prototype succeeds in getting data from a vehicle in operation, permitting its tracking and potential automated analyses over the WWW. 

\noindent {{\bfseries Keywords:} Automotive embedded systems; Vehicular tracking; Internet of Things; Big Data;} 
 \end{otherlanguage*}
\end{resumo}