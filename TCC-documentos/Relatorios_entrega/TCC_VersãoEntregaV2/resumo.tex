%% RESUMOS

%% Resumo em Portugu^es. OBRIGATORIO.
\setlength{\absparsep}{18pt} 
\begin{resumo}

\noindent 
Este trabalho tem o objetivo realizar um estudo a respeito das tecnologias embarcadas presentes nos automóveis para entender os padrões de comunicação entre os sistemas adotados por eles, e como a informática pode contribuir para transparecer as informações e facilitar ou automatizar algumas análises. A fim de aplicar os conceitos estudados, é proposto a implementação de um sistema utilizando um hardware barato, como uma placa de \textit{Raspberry Pi}, que seja capaz de ler as informações provenientes nos sensores dos automóveis e se comunicar com serviços de computação em nuvem. O desenvolvimento deste sistema é dividido em três frentes distintas: a primeira frente é relacionada ao desenvolvimento do software responsável por ler as informações do automóvel; a segunda frente é responsável por preparar o ambiente do \textit{Raspberry Pi} para execução, incluindo a parte de configuração do dispositivo e a implantação do software, e a terceira frente está relacionado com a integração do software com a computação em nuvem. O desenvolvimento do software foi concebido utilizando a linguagem Java com o paradigma de programação orientado à objeto, além dos padrões\textit{ Model-View-Controller (MVC)} e \textit{Data Access Object (DAO)}. Os serviços de computação em nuvem utilizados foram o banco de dados MongoDB e o \textit{framework} Node.js para a criação do web service. A execução do sistema permite gerar dados que ficarão armazenados na nuvem e poderão ser disponibilizados para outras aplicações envolvendo Internet das Coisas e \textit{Big Data}. O resultado final foi obtido com o sistema gerando dados do veículo em funcionamento, permitindo o monitoramento e possíveis análises automatizadas via web.

\vspace{1cm}

\noindent {{\bfseries Palavras-chave:} Sistemas embarcados automotivos; Monitoramento veicular; Internet das Coisas; \textit{Big Data};}

\end{resumo}