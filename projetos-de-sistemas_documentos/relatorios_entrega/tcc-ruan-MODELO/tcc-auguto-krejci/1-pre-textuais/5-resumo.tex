%% RESUMOS

%% Resumo em Portugu^es. OBRIGATORIO.
\setlength{\absparsep}{18pt} 
\begin{resumo}

\noindent 
\hspace{1.5cm} Os desastres naturais são um antigo perigo para o bem estar e a sobrevivência da humanidade. Prever quando ocorrerão, onde, e em que intensidade, é determinante para a mitigação dos riscos, perdas e danos materiais à população. Este trabalho aplica conceitos de inteligência artificial, mais especificamente numa especialização do aprendizado de máquina chamada aprendizado profundo. Propõe a elaboração de um modelo capaz de predizer o nível de uma bacia hidrológica, a partir da informação de sensores de chuva e nível, nela instalados, visando promover alertas a população potencialmente afetada. O modelo utiliza um tipo de algoritmo de aprendizado profundo chamado \textit{Long Short-Time Memory}, que é um tipo de rede neural recorrente bastante adequado a problemas de regressão em séries temporais. O modelo apresentou bons resultados, nas métricas aplicadas, e demonstrou ser um modelo viável.

\vspace{1cm}

\noindent {{\bfseries Palavras-chave:} Predição. Inteligência Artificial. Machine Learning. Deep Learning. Redes Neurais Artificias. LSTM.}

\end{resumo}