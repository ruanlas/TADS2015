\chapter{OBJETIVOS}\label{CAP3}
Esta seção abordará os objetivos gerais e específicos deste trabalho.
\section{OBJETIVO GERAL}
Este trabalho visa buscar um conhecimento amplo de toda a arquitetura automotiva relacionada aos seus sistemas embarcados, e como esses sistemas se comunicam entre si. Além deste estudo, o trabalho propõe também: buscar conhecimento técnico sobre sistemas computadorizados de baixo custo, com a finalidade de poder interagir com a rede veicular interna; adquirir aprendizado sobre a arquitetura, a infraestrutura e o armazenamento web utilizando serviços de computação em nuvem, como a \textit{Amazom Web Services (AWS)}, a fim de manter os dados que foram coletados e enviados através de um dispositivo de baixo custo.

Desta forma, este trabalho propõe a implementação de um sistema computadorizado de baixo custo que se conecte e interaja com a rede interna do automóvel – sendo possível coletar algumas informações presentes nesta rede – e que transmita os dados para um serviço de computação em nuvem com o objetivo de manter essas informações para uma futura análise e ou alguma aplicação envolvendo a Internet das Coisas.

\section{OBJETIVOS ESPECÍFICOS}
Para alcançar o sucesso deste trabalho, o objetivo geral foi dividido em alguns objetivos específicos que estão listados abaixo:
\begin{itemize}
\item Estudar a arquitetura dos sistemas embarcados presentes nos automóveis;
\item Levantar e estudar os protocolos de comunicação utilizados pela rede interna automotiva;
\item Pesquisar por sistemas computadorizados de baixo custo e estudo da arquitetura;
\item Analisar a viabilidade de integração de um sistema computadorizado de baixo custo à rede veicular interna;
\item Desenvolver um software embarcado responsável por realizar a conexão e interação do dispositivo computadorizado com a rede interna do automóvel;
\item Estudar a arquitetura e infraestrutura dos serviços de computação em nuvem;
\item Implementar um banco de dados utilizando uma infraestrutura de computação em nuvem;
\item Integrar o software com o banco de dados para armazenar as informações coletadas;
\item Desenvolver uma página web para consultar e disponibilizar as informações contidas no banco de dados.
\end{itemize}

