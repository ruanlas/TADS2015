\chapter{CONCLUSÕES}\label{CAP7}
Este trabalho mostrou um estudo detalhado da arquitetura presente nos sistemas embarcados automotivos e também a forma com que estes sistemas se comunicam, apresentando as categorias e protocolos utilizados, e como o sistema é dividido. Neste estudo também foi observada a limitação de informações disponibilizadas pelas montadoras sobre os sistemas eletrônicos presentes em seus automóveis. Foi visto também que é fundamental a realização do diagnóstico para a prevenção de eventuais problemas, e a importância do estudo de aplicações voltadas ao cenário automotivo, buscando integrá-lo à tecnologia da informação. Na sequência, foram explorados também alguns conceitos de informática que puderam se relacionar diretamente e indiretamente com o ambiente de um automóvel e como esses conceitos poderiam se cruzar, permitindo aplicações possivelmente viáveis aplicadas à internet das coisas.

Com base neste estudo, foi proposto o desenvolvimento de um sistema responsável por ler as informações relacionadas ao funcionamento do automóvel presentes em uma rede veicular, processar os dados e enviá-los a um banco de dados localizado na nuvem, e a partir disso, disponibilizar estas informações para o acesso de eventuais aplicações, buscando colocar em prática os conceitos estudados ao longo do trabalho. Entretanto, apesar do êxito em efetuar a leitura dos dados presentes nos automóveis, não foi possível verificar se a conversão das informações lidas do automóvel foi precisa por faltar parâmetros de comparação autênticos.

\section{PROPOSTA PARA TRABALHOS FUTUROS}
As dificuldades enfrentadas ao longo do percurso deixam margem para a exploração de outros trabalhos relacionados, além desta proposta. Existem alguns ajustes que podem ser aperfeiçoados numa extensão deste projeto.

Um dos pontos de melhoria é relacionado à tela do \textit{Raspberry Pi} que foi adquirida. Notou-se que a tela de 3.2 polegadas era muito pequena para a exibição das informações, o que levou à adaptação da interface para exibição dos dados. Isso resultou na redução de informações que seriam exibidas na tela. Uma possível sugestão seria a utilização de uma tela de 5 polegadas ou maior.

Outro aspecto que também pode ser explorado é a análise da precisão dos valores que foram obtidos do veículo através do software. É importante comparar os valores gerados pelo sistema com os gerados através de \textit{scanners} e outros equipamentos de leitura originais das montadoras. É importante também analisar os valores que são considerados padrões pelas montadoras junto com as variações permitidas, comparando-as com os valores obtidos.

Analisando-se os resultados referentes ao desempenho do sistema, seria possível considerar a reimplementação do software utilizando a linguagem \textit{Python} para fins de estudo e análise de eficiência do algoritmo. Mesmo levantada a hipótese de uma possível otimização com esta linguagem, é necessário colocar em prática para a análise do resultado e confirmação da teoria.

Uma sugestão de melhoria futura do sistema está relacionado com o estudo de políticas de segurança, uma vez que o sistema está totalmente conectado na web, e em sua concepção não foi medido os possíveis riscos e o impacto que poderia gerar. A questão da privacidade dos dados que serão disponibilizados também é um assunto que poderá ser discutido e definido métricas ou ações a serem tomadas para não infringí-la.

Analisando toda a estrutura do trabalho, é possível notar que o sistema poderá gerar grande volume de dados de forma diversificada (indo ao encontro do conceito definido por \citeauthor{chede} de \textit{Big Data}) tendo uma estrutura totalmente escalável e conectada à internet. Desta forma, observando a infraestrutura utilizada, como o uso de tecnologia de baixo custo, a computação em nuvem e o uso de um banco de dados não relacional, torna-se possível uma extensão de estudo por outras áreas que podem se integrar a este trabalho, como envolvendo aplicações de internet das coisas e \textit{Big Data}.
