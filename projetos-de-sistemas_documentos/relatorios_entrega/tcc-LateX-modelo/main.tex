\documentclass[
        12pt,
        openany, %openright,			
        oneside, %twoside,			%% twoside: para frente e verso ao imprimir
        a4paper,			
        english,			
%	french,				%% Idioma adicional 
%	spanish,			%% Idioma adicional 
        brazil			        %% Idioma principal
        abntfigtabnum     
        ]{abntbibifspcampinas}

\usepackage{lmodern}						
\usepackage[T1]{fontenc}		
\usepackage[utf8]{inputenc}		%% Para converter automaticamente acentos como digitados. Mude utf8 para latin1 se precisar. 
                                        %% Permite digitar os acentos no teclado normalmente, sem comandos (\'e \`a , etc.).
\usepackage[brazil]{babel}
\usepackage{lastpage}	
\usepackage{indentfirst}	
\usepackage{graphicx}			
\usepackage{microtype}
\usepackage{setspace}
\usepackage{color}
\usepackage[abnt-etal-list=0,abnt-etal-text=it,abnt-and-type=&,abnt-emphasize=bf,abnt-full-initials=yes,alf,bibjustif]{abntcite}
\usepackage{multirow}
\usepackage{pbox}
\usepackage{etex}
\usepackage{mathptmx}
\usepackage{times}

\makeatletter
\let\c@lofdepth\relax
\let\c@lotdepth\relax
\makeatother 
\usepackage{subfigure}
\usepackage[colorinlistoftodos]{todonotes}

%\usepackage{fancyhdr}
%\fancypagestyle{plain}{%redefining plain pagestyle
%\fancyhf %clear all headers and footers fields
%\fancyhead[R]{\thepage} %prints the page number on the right side of the header
%}

\usepackage{scrpage2}
\ifoot[]{}
\cfoot[]{}
\ofoot[\pagemark]{\pagemark}

\pagestyle{scrplain} 

%% -----------------------------------------------------------------------------

%% Obs.: Alguns acentos foram omitidos.

\autor{TIAGO JOSÉ DE CARVALHO}
\autorR{de Carvalho, Tiago José} %%Colocar o sobrenome do autor antes do primeiro nome do autor, separados por ,
\titulo{TESTE DE TITULO DE TRABALHO FINAL DE CONCLUSAO DE CURSO NO LATEX} %%Por exemplo, Titulo da tese
% \subtitulo{: subt\'itulo do trabalho}  %% Retirar o primeiro ``%'' desta linha se for utilizar subtitulo. Deixar os dois pontos antes, em ``: subt\'itulo'' . 
\local{CAMPINAS}
\data{2017} %%Alterar o ano se precisar
\orientador[Orientador:]{Tiago José de Carvalho} %%Se precisar, troque [Orientador:] por [Orientadora:]
% \coorientador[Coorientador:]{Nome do coorientador } %% Retirar o primeiro ``%'' desta linha se tiver coorientador. Se precisar, troque por [Cooorientadora:]. 
\instituicao{INSTITUTO FEDERAL DE EDUCAÇÃO, CIÊNCIA E TECNOLOGIA DE SÃO PAULO}
\faculdade{CÂMPUS CAMPINAS} %%Alterar, dentro de chaves {}, se precisar.
\objeto{Trabalho Final de Curso}  %%Dissertacao (Mestrado) %%Tese (Doutorado)
\natureza{Trabalho Final de Curso submetido à banca examinadora constituída de acordo com o Artigo 9$^o$ do Capítulo IV das Normas de Trabalho Final de Curso estabelecidas pelo Colegiado do Curso de \insereprograma,como parte dos requisitos necessários para a obtenção do grau de Engenheiro Civil} 


%% Abaixo, prencher com os dados da parte final da ficha catalografica

\finalcatalog{1. Palavra-chave. 2. Palavra-chave. 3. Palavra-chave. I. Sobrenome, Nome do orientador, orient. II. T\'itulo.} %% Aqui fica 
% escrito a palavra ``T\'itulo'' mesmo, nao o do trabalho. Se tiver coorientador, os dados ficam depois dos dados 
%% do orientador (II. Sobrenome, Nome do coorientador, coorient.) e antes de ``II. T\'itulo'', o qual passa a ``III. T\'itulo''.

%% ---

\setlength{\parindent}{1.3cm}

\setlength{\parskip}{0.2cm}  

\setlength\afterchapskip{12pt}  

\linespread{1.3}

%% Iniciar o documento
\begin{document}

\pagestyle{empty}


%% ELEMENTOS PRE-TEXTUAIS

%% Capa
\inserecapa

%% Contra Capa
% \inserecontracapa

%% Folha de rosto
\inserefolhaderosto

%% Ficha Catalografica
\inserecatalog

%% Folha de aprovacao
\begin{folhadeaprovacao}
\linespread{1.5}

  \begin{center}
    {\chapterfont \insereautor}
    \vfill\vspace{2cm}
    {\chapterfont\bfseries \inseretitulo}
    \vfill\vspace{1.5cm}
    \end{center}
    
    \hspace{.45\textwidth}
    \begin{minipage}{.47\linewidth}
	\vfill	
	Trabalho de Conclus\~{a}o de Curso apresentado como exig\^encia parcial para obten\c{c}\~{a}o do diploma do Curso de Tecnologia em An\'{a}lise e Desenvolvimento de Sistemas do Instituto Federal de Educa\c{c}\~{a}o, Ci\^{e}ncia e Tecnologia C\^{a}mpus Campinas.
    \end{minipage}
\vfill\vspace{1cm}  

\begin{center}
Aprovado pela banca examinadora em: 04 de dezembro de 2017
\vfill \vspace{1cm}
{\large BANCA EXAMINADORA}
   \assinatura{Prof. Dr. \insereorientador \hspace{0.1cm}(orientador)}{IFSP Câmpus Campinas} 
   \assinatura{Prof. Me. André Willik Valenti}{IFSP Câmpus Campinas} 
   \assinatura{Prof. Dr. Andreiwid Sheffer Correa}{IFSP Câmpus Campinas} 
\end{center}
\end{folhadeaprovacao}


% Dedicatoria (opicional)
\begin{flushright}
\begin{minipage}[r]{10cm}
\vspace{18cm}
\textit{
``Dedico este trabalho aos meus familiares, colegas de classe, professores e servidores do Instituto que colaboraram em minha jornada formativa''.
}
\end{minipage}
\end{flushright}


%% Agradecimentos. OPCIONAL. CASO SEJA BOLSISTA, INSERIR OS DEVIDOS AGRADECIMENTOS.
\begin{agradecimentos}

Agradeço primeiramente a Deus, que me proveu de todas as ferramentas necessárias para que eu alcançasse meus objetivos.

À minha mãe, Fabiana, exemplo de vida e meu porto seguro nos momentos difícies. Ao meu padrasto Patrick por todo o apoio. Aos meus irmãos Isabela e Matheus, cujo carinho e amor foram fundamentais. Ao meu padrinho, Fernando César e minha tia, Flávia, por sempre acreditarem no meu potencial. À minha avó, Ana, que tanto fez por mim e ao meu avô José, que sempre levarei no coração.

Aos meus amigos de turma, do PET e de Brunel, que me deram o incentivo pra persistir apesar das adversidades e com quem compartilhei momentos maravilhosos ao longo desses anos. Agradeço especialmente ao Renan, que esteve comigo desde o princípio e à Thais, sua amizade ao longo do último ano foi muito importante pra mim. Às minhas amigas de república, Mariana, Mariane e Thairine, por serem como uma família. E ao Victor, jamais poderei agradecer devidamente pelo seu carinho. 

À aluna Gisele, ao meu orientador Leonardo Goliatt e meu coorientador Geraldo Marques, pela sua ajuda e paciência durante a elaboração deste trabalho. Às professoras Flávia Bastos e Michèle Farage por compartilharem seus conhecimentos além da sala de aula. E a todos os demais professores, essenciais para minha formação.

\end{agradecimentos}
\begin{flushright}
\begin{minipage}[r]{10cm}
\vspace{18cm}
``A mente que se abre a uma nova ideia jamais voltará ao seu tamanho original''.
\begin{flushright}
Albert Einstein
\end{flushright}
\end{minipage}
\end{flushright}
%% RESUMOS

%% Resumo em Portugu^es. OBRIGATORIO.
\setlength{\absparsep}{18pt} 
\begin{resumo}

\noindent 
A caracterização das superfícies de pavimentos mostra-se de extrema importância para garantir a segurança dos usuários e melhor funcionamento dos veículos nas rodovias. O desempenho dos pavimentos é avaliado e classificado a partir de vários parâmetros, sendo a rugosidade um dos principais, pois influencia diretamente a qualidade do contato entre pneu e asfalto. Atualmente, tal caracterização é feita por meio de ensaios indiretos, como o Ensaio de Mancha de Areia, que apresentam um alto nível de imprecisão; ou por meio de equipamentos de medição denominados perfilômetros, os quais apresentam um alto custo e cuja utilização é bastante limitada. Dessa forma, notou-se a necessidade de desenvolver métodos alternativos que sejam eficientes. O presente trabalho apresenta um método desenvolvido com base na tecnologia LiDAR (\emph{Light Detection and Ranging}), a qual utiliza-se de equipamentos de escaneamento tridimensional a laser para obter um modelo de um objeto. O método consiste em escanear a superfície de um corpo de prova extraído do pavimento e seccioná-lo por sucessivos planos transversais, obtendo-se perfis representativos da superfície. Posteriormente, o resultado é analisado por um algoritmo que realiza o cálculo da altura média dos perfis, resultando em um parâmetro que pode ser associado à medida de textura superficial obtida em campo. O emprego desta metodologia mostra-se promissor, as alturas de areia encontradas para os pavimentos de concreto e de asfalto foram respectivamente 1,000mm e 1,035mm, sendo condizentes com a análise visual; além disso o coeficiente de determinação $(R^2)$, que relaciona os valores experimentais e computacionais foi de 0,9815, indicando uma alta precisão.

\vspace{1cm}

\noindent {{\bfseries Palavras-chave:} pavimento asfáltico; rugosidade; textura superficial; irregularidade; escaneamento tridimensional; mancha de areia;}

\end{resumo}
%% Resumo em Ingle^s
\begin{resumo}[ABSTRACT]
 \begin{otherlanguage*}{english}
   This project aims at studying embedded technologies found in cars in order to understand the communication patterns among the systems adopted by the automotive industry. Also, intends to explore means by which computer science can contribute to make clear the information involved and simplify or automate some analyses. In order to apply the concepts found, it was proposed the implementation of a system upon a low cost hardware platform (using a Raspberry Pi board) able to read information originated from cars’ sensors and to communicate with cloud computing services. This systems development was divided in three distinct fronts. The first one is related to the software in charge of obtain the vehicle's data; the second is responsible for prepare the Raspberry Pi's environment to execute, including the device configuration and software installation. The last one deals with the integration between de embedded software and the cloud computing platform. The whole development was designed under the object oriented paradigm, using Java, as well as architectural patterns Model-View-Controller (MVC) and Data Access Object (DAO). The cloud computing services adopted have used MongoDB as database management system and the Node.js framework for web service creation. The system allows generating data to be stored in the cloud and which can be made available to other applications involving diverse areas as, for example, Internet of Things and Big Data. The final prototype succeeds in getting data from a vehicle in operation, permitting its tracking and potential automated analyses over the WWW. 

\noindent {{\bfseries Keywords:} Automotive embedded systems. Vehicular tracking. Internet of things. Big Data.} 
 \end{otherlanguage*}
\end{resumo} 
 

%% Seguindo o mesmo modelo acima, pode-se inserir resumos em outras linguas. 

%% Lista de ilustracoes. OPCIONAL.
\pdfbookmark[0]{\listfigurename}{lof}
\listoffigures*
\cleardoublepage


%% Lista de tabelas. OPCIONAL. Retire o ``%'' de cada das 3 linhas seguintes, caso queira.
\pdfbookmark[0]{\listtablename}{lot}
\listoftables*
\thispagestyle{empty}
\cleardoublepage


%% Lista de abreviaturas. OPCIONAL
\begin{abreviaturas} %%ALTERAR OS EXEMPLOS ABAIXO, CONFORME A NECESSIDADE
\item[{abr.}]\emph{abril}%
\item[{adapt.}]\emph{adaptação} %
\end{abreviaturas}




%% Lista de siglas. OPCIONAL
\begin{siglas} %%ALTERAR OS EXEMPLOS ABAIXO, CONFORME A NECESSIDADE
\item[{ABNT}]\emph{Associação Brasileira de Normas Técnicas}%
\item[{ALS}]\emph{Airborne Laser Scanning} %
\item[{ASTM}]\emph{American Society for Testing and Materials}%
\item[{CA}] Concreto Asfáltico%
\item[{CAL}] Coeficiente de Atrito Longitudinal%
\item[{CBUQ}] Concreto Betuminoso Usinado à Quente%
\item[{CNT}] Confederação Nacional do Transporte%
\item[{CPA}] Camada Porosa de Atrito%
\item[{DNER}] Departamento Nacional de Estradas de Rodagem%
\item[{DNIT}] Departamento Nacional de Infraestrutura de Transportes%
\item[{HS}] \emph{Height of Sand}%
\item[{INIR}] Instituto de Infra-Estruturas Rodoviárias%
\item[{IPR}] Instituto de Pesquisas Rodoviárias%
\item[{IRI}]\emph{International Roughness Index}%
\item[{ISO}]\emph{International Organization for Standardization}%
\item[{LiDAR}]\emph{Light Detection and Ranging}%
\item[{MLS}]\emph{Mobile Laser Scanning}%
\item[{MPD}]\emph{Mean Profile Depth}%
\item[{MTD}]\emph{Mean Texture Depth Test}%
\item[{MLT}]\emph{Multistripe Laser Triangulation}% 
\item[{PCA}]\emph{Análise de Componentes Principais}%
\item[{PIARC}]\emph{Permanent International Association of Road Congresses}%
\item[{SMA}]\emph{Stone Mastic Asphalt}%
\item[{TLS}]\emph{Terrestrial Laser Scanning}%
\item[{USP}] Universidade de São Paulo%
\item[{VRD}] Valor de Resistência à Derrapagem%
\end{siglas}

%% Lista de simbolos. OPCIONAL
\begin{simbolos} %%ALTERAR OS EXEMPLOS ABAIXO, CONFORME A NECESSIDADE
 \item[$Zt$] altura de um elemento%
\item[$Rp$] altura máxima do pico%
\item[$Zp$] altura de pico%
\item[$Rt$] amplitude do perfil%
\item[$R^2$] coeficiente de determinação
\item[$L$] comprimento de amostragem%
\item[$Rms$] desvio quadrático médio%
\item[$Rku$] fator de achatamento ou curtose%
\item[$LM$] linha média%
\item[$Md$] mediana%
\item[$Rv$] profundidade máxima do vale%
\item[$Zv$] profundidade de vale%
\item[$Ra$] rugosidade média ou amplitude média%
\item[$MPD$] textura média do perfil%
 \end{simbolos}

\pagestyle{empty}

%% Sumario
\pdfbookmark[0]{\contentsname}{toc}
\tableofcontents
\cleardoublepage

%% ----------------------------------------------------------

%% ELEMENTOS TEXTUAIS

\textual
%\pagestyle{plain}
\pagestyle{scrplain} 
\clearscrheadfoot            %<---
\rohead[\pagemark]{\pagemark}%<---
\chapter{INTRODUÇÃO}\label{CAP:introducao}
%\thispagestyle{empty}

\section{CONTEXTUALIZAÇÃO}
Existem na atualidade diversos tipos e modelos de automóveis possuindo diversas tecnologias integradas a eles. Entretanto, nem sempre foi assim. No início da história automotiva, surgem os primeiros carros a manivela em meados de 1880 e logo após chegam os carros a combustão interna. Depois de um tempo surgiram os carros carburados e depois de um certo período, chegaram os carros com injeção eletrônica. É observado que a cada período que se passa, o automóvel ganha alguns itens e tecnologias novas com a finalidade de melhorar o desempenho, combinando baixo consumo e baixa emissão de poluentes. Independente do ano ou modelo do veículo, ou até mesmo a tecnologia adotada por ele, o desgaste natural das peças é comum a qualquer modelo de automóvel. Vale ressaltar que estes desgastes podem ser agravados dependendo da forma com que o motorista utiliza seu veículo. Independente da causa do desgaste, quando há algum problema presente, geralmente o automóvel passa a apresentar um comportamento fora do comum, o que indica uma possível falha. Quando esses comportamentos são emitidos na forma de ruídos, há uma certa facilidade na percepção de que algo está errado, e uma possível manutenção ou revisão deve ser feita. Porém, quando esses comportamentos não fazem emissão de nenhum sinal ou ruído aparente, existe uma certa dificuldade na percepção de algum eventual defeito.

Os veículos que circulam atualmente possuem diversos sensores e controladores que fazem parte de um sistema embarcado automotivo. Esses sensores são responsáveis basicamente por coletar algumas informações do veículo e automatizar algum funcionamento específico do automóvel. Um exemplo típico, é o controle de injeção de combustível, que é feito de forma eletrônica na maioria dos carros que trafegam pelas cidades e estradas. Este, por exemplo, evita o consumo excessivo de combustível durante um trajeto percorrido. Outra situação comum se encontra no painel de instruções, onde são mostradas algumas informações limitadas que são monitorados pelos sensores e exibidas ao condutor, como a rotação do motor, indicador de velocidade, temperatura do motor, entre outros. 

Analisando os veículos atuais, nota-se que parte de seu funcionamento está deixando de ser apenas mecânico, e passando a ser controlado por sistemas eletrônicos. Estes sistemas, segundo \citeonline{smith}, podem ser considerados também como dispositivos informatizados por possuírem capacidade de processamento. Ele ainda reforça que a tecnologia presente nos automóveis está tendendo mais à complexidade e à conectividade. Essa tendência atrelada à conectividade ressaltada pelo autor vai ao encontro do conceito de Internet das Coisas \textit{(Internet of Things – IoT)}, dando potencialidade a estudos de aplicações explorando o tema. Segundo o projeto \textit{Coordination And Support Action for Global RFID-related Activities and Standardisation \cite{casagras}}, o conceito de Internet das Coisas se refere a uma infraestrutura de rede global capaz de interligar objetos físicos e virtuais através da exploração da capacidade de capturar de dados e de se comunicarem entre si.

\section{JUSTIFICATIVA}
De acordo com o que foi apresentado até o momento, é nitidamente visível que existe uma complexidade nos sistemas presentes nos automóveis, e isso se justifica com a evolução tecnológica e a informatização automotiva. Devido a este fator dominante, a percepção de algum eventual defeito em qualquer um desses conjuntos eletrônicos do veículo é uma tarefa complexa e dificilmente visível ao condutor.

Baseado nestas informações, é importante explorar estes conceitos para entender a relação desses sistemas embarcados automotivos com a informática, e como a tecnologia da informação pode contribuir para alavancar o crescimento desta área, podendo proporcionar conhecimento amplo destes sistemas, desde a arquitetura de operação, comunicação e processamento de dados veiculares, a fim de facilitar o entendimento de possíveis defeitos ou anomalias eletrônicas, além de estudar propostas de monitoramento e diagnóstico que podem facilitar o entendimento destes sistemas.

\section{DEFINIÇÃO DO PROBLEMA}
Analisando os fatos apresentados, identifica-se certo distanciamento do condutor com os eventuais problemas eletrônicos que os veículos estão sujeitos à apresentarem. Esse distanciamento é justificado parcialmente devido ao sistema apresentar alta complexidade, e também pelo fato dos problemas não emitirem sinais aparentes sinalizando algum defeito. Essa distância é aumentada à medida que que o automóvel imerge na informática, automatizando parte de seu funcionamento fazendo o uso de sensores e sistemas embarcados.

\section{OBJETIVOS}
Baseado na contextualização e na definição do problema, este trabalho propõe a exploração e estudo aprofundado de como funciona a arquitetura e comunicação dos sistemas embarcados presentes nos automóveis. A partir do entendimento desta proposta, o objetivo será estudar a viabilidade de implementação de um sistema computadorizado de baixo custo para interagir com essa rede veicular interna, seguindo a tendência de uma aplicação para \textit{IoT}.

\section{AVALIAÇÃO DOS RESULTADOS}
O sucesso deste trabalho será medido de acordo com o cumprimento dos objetivos propostos, buscando estudar e aplicar conceitos que serão abordados durante o desenvolvimento desta pesquisa.

\section{TRABALHOS RELACIONADOS}
Foram pesquisados alguns trabalhos com propostas semelhantes a fim de entender algumas dificuldades enfrentadas e propor algumas melhorias. De todos os trabalhos coletados, foram analisadas três propostas relacionadas ao tema deste.

A primeira obra analisada foi proposta por \citeonline{marques}, que apresentava o desenvolvimento de um sistema composto por hardware e software em tempo real para a comunicação com a rede \textit{CAN} dos sistemas automotivos. O trabalho era voltado para análise e diagnóstico destes sistemas veiculares.

O segundo trabalho foi apresentado por \citeonline{fagundesetall}, que se propuseram em coletar os dados da interface \textit{OBD-II} e trata-los utilizando um microcontrolador, para assim poder enviar estas informações utilizando uma rede GSM. Estas informações seriam lidas através de um browser de internet em um computador.

O terceiro trabalho considerado de maior relevância foi proposto por \citeonline{staroski}, que tinha por objetivo estudar a viabilidade de desenvolvimento de um protótipo de software embarcado em uma placa de \textit{Raspberry Pi} para monitorar os sensores presentes em um automóvel. Com este protótipo seria possível o monitoramento via web em tempo real do veículo.

Entretanto, o trabalho apresentado por \citeonline{staroski} apresenta alguns pontos que podem ser melhorados e que serão explorados durante a elaboração desta proposta.


%\section{ORGANIZAÇÃO DO TRABALHO}

%O Capítulo \ref{CAP2} traz uma revisão de literatura, abordando a definição de alguns conceitos essenciais para viabilizar o entendimento do método e a descrição de estudos anteriores pertinentes ao tema.

%O Capítulo  \ref{CAP3} apresenta os materiais e métodos utilizados neste trabalho, são descritas as especificações técnicas dos equipamentos utilizados, além da descrição dos ensaios realizados; 

%O Capítulo  \ref{CAP4} apresenta os resultados dos métodos experimental e computacional e realiza uma comparação entre eles. São calculadas as variações entre os resultados como forma de verificar sua precisão; 

%O Capítulo \ref{CAP5} traz as conclusões inferidas a partir dos resultados obtidos e analisa possibilidades de expansão deste tema em futuros trabalhos.



\chapter{JUSTIFICATIVA}\label{CAP2}

De acordo com o que foi abordado na introdução, sabe-se que o desgaste das peças de um automóvel é natural e inevitável. Entretanto, ter a possibilidade de prever o desgaste fazendo uma análise geral do estado de cada item que compõe o veículo pode ser uma alternativa satisfatória. Quando alguma peça começa a apresentar algum defeito e não é tratado com o devido cuidado, pode acarretar em problemas maiores, levando o condutor a ter gastos excessivos com a manutenção.

A evolução da tecnologia no cenário automobilístico traz várias melhorias, visando ao conforto e segurança do condutor, confiabilidade nos sistemas e otimização de consumo. Entretanto, por conter certa complexidade, acaba dificultando a percepção de algum desgaste ou falha de alguma peça ou dispositivo específico. Percebe-se aqui o resultado da informatização dos sistemas automotores. A indústria automotiva, segundo \citeonline{smith}, tem criado veículos com sistemas eletrônicos de alta complexidade, mas disponibilizou poucas informações sobre como esses sistemas funcionam.

Percebe-se aqui uma abstração de funcionamento muito grande que ocorre dentro dos sistemas embarcados veiculares. Essa abstração dificulta a detecção de falhas justamente por não emitir sinais aparentes. O autor também reforça que normalmente esses sistemas eletrônicos automotivos são normalmente fechados, com exceção somente para a oficina mecânica da concessionária.

Voltando à contextualização sobre o fato da imersão automobilística no mundo informatizado, segundo o relatório de \citeonline{charette}, publicado no \textit{IEEE Spectrum}, ele observa que existem de 70 a 100 microprocessadores integrados em unidades de controle eletrônico \textit{(electronic control units – ECUs)} e que são capazes de executar cerca de 100 milhões de linhas de código de software. \citeonline{smith} ainda reforça afirmando que à medida que os sistemas informáticos se tornam mais integrantes dos veículos, a realização de avaliações de segurança torna-se mais importante e complexa.

Baseado nessas informações e considerando a tendência tecnológica dos veículos, é importante estudar e aplicar os conceitos de tecnologias de baixo custo para a interação com a rede automotiva além das possíveis aplicações envolvendo a Internet das Coisas para auxiliar no diagnóstico e monitoramento automotivo. 
\chapter{OBJETIVOS}\label{CAP3}
Esta seção abordará os objetivos gerais e específicos deste trabalho.
\section{OBJETIVO GERAL}
Este trabalho visa buscar um conhecimento amplo de toda a arquitetura automotiva relacionada aos seus sistemas embarcados, e como esses sistemas se comunicam entre si. Além deste estudo, o trabalho propõe também: buscar conhecimento técnico sobre sistemas computadorizados de baixo custo, com a finalidade de poder interagir com a rede veicular interna; adquirir aprendizado sobre a arquitetura, a infraestrutura e o armazenamento web utilizando serviços de computação em nuvem, como a \textit{Amazom Web Services (AWS)}, a fim de manter os dados que foram coletados e enviados através de um dispositivo de baixo custo.

Desta forma, este trabalho propõe a implementação de um sistema computadorizado de baixo custo que se conecte e interaja com a rede interna do automóvel – sendo possível coletar algumas informações presentes nesta rede – e que transmita os dados para um serviço de computação em nuvem com o objetivo de manter essas informações para uma futura análise e ou alguma aplicação envolvendo a Internet das Coisas.

\section{OBJETIVOS ESPECÍFICOS}
Para alcançar o sucesso deste trabalho, o objetivo geral foi dividido em alguns objetivos específicos que estão listados abaixo:
\begin{itemize}
\item Estudar a arquitetura dos sistemas embarcados presentes nos automóveis;
\item Levantar e estudar os protocolos de comunicação utilizados pela rede interna automotiva;
\item Pesquisar por sistemas computadorizados de baixo custo e estudo da arquitetura;
\item Analisar a viabilidade de integração de um sistema computadorizado de baixo custo à rede veicular interna;
\item Desenvolver um software embarcado responsável por realizar a conexão e interação do dispositivo computadorizado com a rede interna do automóvel;
\item Estudar a arquitetura e infraestrutura dos serviços de computação em nuvem;
\item Implementar um banco de dados utilizando uma infraestrutura de computação em nuvem;
\item Integrar o software com o banco de dados para armazenar as informações coletadas;
\item Desenvolver uma página web para consultar e disponibilizar as informações contidas no banco de dados.
\end{itemize}


\chapter{FUNDAMENTAÇÃO TEÓRICA}\label{CAP4}
Nesta seção serão abordados todos os assuntos que foram estudados para a contrução deste trabalho.
\section{SISTEMAS E SOFTWARE EMBARCADOS}\label{secaosistemasembarcados}
A definição de sistemas embarcados, segundo \citeonline{leeseshia}, são sistemas computacionais pouco perceptíveis, que geralmente são responsáveis por executar pequenas atividades de forma autônoma, como controlar os robôs da linha de produção de uma fábrica ou gerenciar os semáforos de uma cidade. De acordo com \citeonline{carrowagner} os sistemas computacionais embarcados estão presentes em boa parte das atividades humanas, passando desde o sistema de transporte até os eletrodomésticos de uma residência. Baseado nestes argumentos, sistemas embarcados são todos os dispositivos com poder de processamento, memória e fontes de energia limitados \cite{leeseshia}, que podem se integrar com o meio físico.

\citeonline{leeseshia} ainda reforçam que os programas que são executados nestes dispositivos são chamados de software embarcado. \apudonline{gill}{leeseshia} da \textit{National Science Foundation in the US} cria o termo sistemas ciberfísicos \textit{(Cyber-Physical Systems – CPS)} para se referir à integração da computação com processos físicos. Observa-se aqui uma outra definição para sistemas embarcados. No \textit{CPS}, os sistemas informatizados monitoram e controlam os processos físicos geralmente executando instruções dentro de loops. \citeonline{leeseshia} ainda reforça que é importante compreender a dinâmica dos sistemas computacionais junto com os processos físicos. Por lidar diretamente com o mundo físico, o tempo necessário para executar uma tarefa, nos sistemas ciberfísicos, pode ser fundamental para o correto funcionamento do sistema. A passagem do tempo no mundo físico é algo crítico, ao contrário do mundo cibernético.

Enquanto no processo físico existem diversas coisas acontecendo concorrentemente (ao mesmo tempo), nos processos de software as atividades acontecem em etapas sequenciais. \citeonline{leeseshia} afirmam que o maior desafio técnico na concepção e análise do software embarcado se deriva da necessidade de unir a semântica sequencial do mundo lógico com a realidade concorrente do mundo físico.

\section{SISTEMAS DISTRIBUÍDOS}\label{secaosistemasdistribuidos}
Um sistema distribuído, segundo \citeonline{tanenbaum}, é um conjunto de computadores independentes que que se apresentam ao usuário final como um único sistema coerente. De maneira genérica, a arquitetura de um sistema distribuído é composto por diversos itens de hardware que atuam de forma autônoma - geralmente processando informações específicas - mas que se comunicam entre si caracterizando-se como apenas um único hardware responsável pelo sistema como um todo. O autor ainda reforça que as pessoas ou usuários acham que estão interagindo com um sistema apenas, e não com um conjunto de sistemas. Contudo, para garantir a operação de todo este conjunto, é preciso que haja a colaboração de todos os componentes e dispositivos que fazem parte do sistema. A essência do sistema distribuído está em estabelecer essa colaboração.

Não é estabelecido nenhum padrão ou premissa relacionado ao tipo de hardware que irá compor o sistema. \citeonline{tanenbaum} ainda afirma que esses dispositivos podem variar desde computadores centrais até pequenos nós em redes de sensores. Também não existe premissa com relação ao modo com que os dispositivos se interligam. A característica principal dos sistemas distribuídos está em ocultar boa parte da comunicação interna deste conjunto ao usuário. \citeonline{tanenbaum} define ainda quatro metas que devem ser cumpridas para que o esforço necessário para a construção de um sistema distribuído seja válida: o sistema deve oferecer fácil acesso a seus recursos; deve ocultar razoavelmente bem o fato de que os recursos são distribuídos por uma rede; deve ser aberto e deve permitir a sua expansão.

\section{ARQUITETURA DO SISTEMA AUTOMOTIVO}
Para compreender como funciona a comunicação interna do veículo, é necessário antes saber como o sistema está estruturado, quais dispositivos e controladores utilizados, qual a arquitetura adotada e quais protocolos estão implementados. Para \citeonline{navetsimonotlion}, os fabricantes de automóveis diferenciam em várias categorias os eletrônicos embarcados que um carro possui. Eles utilizam essas categorias para agrupar sistemas mecânicos e ou eletrônicos de acordo com as suas funcionalidades.

\subsection{\textbf{Categorias funcionais dos sistemas embarcados automotivos}}
Historicamente, segundo os autores, existem cinco categorias de sistemas embarcados: \textit{Power Train}, \textit{Chassis}, \textit{Body}, \textit{HMI} e \textit{Telematics}. A categoria \textit{Power Train} fazem parte todos os sistemas que participam da propulsão longitudinal do veículo, incluindo o motor, a transmissão e todos os componentes que dão apoio para esta função. A categoria \textit{Chassis} se refere às quatro rodas e à sua posição relativa de movimento. Nesta categoria os principais sistemas são o de freios e direção. Dentro da categoria \textit{Body} estão presentes as entidades que não pertencem à dinâmica do veículo, mas que auxiliam o motorista, como o airbag, limpadores, iluminação, vidros, ar condicionado, assentos, etc. Já a categoria \textit{HMI} inclui o equipamento que permite a troca de informações entre os sistemas eletrônicos e o motorista do veículo. Por fim, a categoria \textit{Telematics} está relacionado à componentes que permitem a troca de informações do veículo com o mundo exterior, como rádio, sistemas de navegação, GPS, entre outros. \citeonline{navetsimonotlion} ainda ressaltam que cada categoria do sistema eletrônico possui características bem diferentes, o que faz com que cada dispositivo tenha um requisito ou uma restrição bem definida. A Figura \ref{Fig:categorias_sistemas_embarcados} ilustra a divisão funcional dos sistemas pelas categorias.

\begin{figure}[!ht]
\centering
\caption{Representação das cinco categorias funcionais.} 
{\includegraphics[scale=.31]{imagens/categoriaFuncionalSistemasEmbarcados.png}}\\
\makebox[\width]{Fonte: baseado em \citeonline{navetsimonotlion}} \label{Fig:categorias_sistemas_embarcados}
\end{figure}


\begin{itemize}
\item{Categoria \textit{Power Train}:}
Além de controlar a velocidade do motor, atuando de acordo com as intervenções do motorista no pedal, o controlador pode também, atuar de acordo com fatores naturais, como a temperatura do ar ou o nível de oxigênio, ou atuar de acordo com os distúrbios ambientais, como a poluição dos gases de escape ou o ruído. Ainda segundo \citeonline{navetsimonotlion}, o controlador é projetado para otimizar alguns parâmetros. O parâmetro mais comum a ser controlado é a quantidade de combustível que deve ser injetado para combustão de acordo com a rotação do motor e a posição do pedal do acelerador. Observa-se que nesta categoria estão presentes todos os dispositivos que auxiliam no gerenciamento do motor, tanto de forma direta quanto indireta.

\item{Categoria \textit{Chassis}:}
Nesta categoria, existem sistemas responsáveis por gerenciar a interação do veículo com a estrada. O objetivo destes sistemas é controlar o automóvel de acordo com as solicitações do motorista, como frenagem ou aceleração, considerando também o perfil da via ou as condições ambientais, visando sempre o conforto e a segurança dos passageiros. \citeonline{navetsimonotlion} ainda reforça que esses sistemas devem ser de alta qualidade, como qualquer sistema crítico. Os sistemas mais comuns são o de frenagem (ABS) e o controle automático de estabilidade (ASC).

\item{Categoria \textit{Body}:}
Limpadores de para-brisa, luzes, portas e janelas e outros itens são controlados por sistemas pertencentes a esta categoria. Estes, por sua vez, não estão sujeitos a restrições de desempenho rigorosas, e do pondo de vista de segurança, segundo os autores, não representam uma parte crítica do sistema. Entretanto, existem certas funções, como controlar o acesso ao veículo, que deve respeitar as dificuldades em tempo real.

\item{Categoria \textit{HMI}:}
Os sistemas presentes nesta categoria, em geral, permite a interação do motorista com diversas funções integradas no veículo. Uma das funções é exibir o estado atual do veículo, como velocidade, rotação do motor e temperatura, por exemplo, ou também mostrar o estado de algum dispositivo multimídia.

\item{Categoria \textit{Telematics}:}
Esta categoria, ainda segundo \citeonline{navetsimonotlion}, inclui sistemas que suportam trocas de informações entre infra-estruturas viárias e rodoviárias. Um exemplo está relacionado com a cobrança automática de pedágios. Para os autores, em um futuro próximo, esta categoria permitirá otimizar o uso rodoviário através da gestão do tráfego a fim de evitar congestionamentos.
\end{itemize}

A divisão dos sistemas por categoria engloba dispositivos de acordo com suas semelhanças funcionais. Logo, se um conjunto eletrônico, mesmo que sejam diferentes, seguirem um determinado requisito comum à uma categoria, estes pertencerão à esta. Entretanto, os sistemas eletrônicos, mesmo pertencendo à uma categoria distinta, não são impedidos de se comunicarem entre si. Segundo o exemplo de \citeonline{navetsimonotlion}, as informações como a rotação do motor, temperatura ou a velocidade fazem parte do gerenciamento do motor, e logo pertencem à categoria \textit{power train}, mas são transmitidas desta para a categoria \textit{HMI}, para exibir as informações do veículo ao motorista no painel de instruções (Figura \ref{Fig:comunicacao_categorias_sistemas_embarcados}).

\begin{figure}[!ht]
\centering
\caption{Representação do monitoramento de RPM e a comunicação entre as categorias.} 
{\includegraphics[scale=.37]{imagens/intercomunicacaoEntreCategorias.png}}\\
\makebox[\width]{Fonte: baseado em \citeonline{navetsimonotlion}} \label{Fig:comunicacao_categorias_sistemas_embarcados}
\end{figure}

Analisando o funcionamento destes sistemas eletrônicos, eles são caracterizados como sistemas embarcados automotivos pelo fato de possuírem uma forte interação com o mundo físico, conforme a definição de sistemas embarcados explorado na seção \ref{secaosistemasembarcados}. Para interagir com o mundo físico, estes sistemas fazem o uso de outros dispositivos conhecidos como sensores e atuadores. Segundo \citeonline{leeseshia}, um sensor mede uma quantidade física, enquanto o atuador altera a quantidade física. Eles ainda complementam que estes dispositivos conectam o mundo cibernético com o mundo físico. Em outras palavras, um sensor obtém os dados de uma leitura, e o atuador realiza uma ação que foi passado a ele. Em um automóvel existem diversos sensores espalhados por sua estrutura e que monitoram diversos aspectos físicos do veículo, como aceleração lateral, velocidade e tração individual das rodas, monitorando a estabilidade do carro durante o percurso \cite{navetsimonotlion}. O autor ainda complementa que quando uma correção precisa ser aplicada nesta situação, as rodas dianteiras ou traseiras podem frear individualmente e com intensidades diferentes, ou atuar na redução da potência do motor. Percebe-se neste exemplo a ação dos atuadores no meio físico. A Figura \ref{Fig:relacao_sensor_atuador} representa a influência destes dispositivos no meio físico.

\begin{figure}[!ht]
\centering
\caption{Representação da atuação dos sensores e atuadores.} 
{\includegraphics[scale=.29]{imagens/relacaoSensorAtuador.png}}\\
\makebox[\width]{Fonte: baseado em \citeonline{leeseshia}} \label{Fig:relacao_sensor_atuador}
\end{figure}

Analisando-se também a forma como esses dispositivos trocam informações uns com os outros, percebe-se que está presente o conceito de sistemas distribuídos, conforme apresentado na seção \ref{secaosistemasdistribuidos}. Um exemplo desse conceito no cenário automotivo é mencionado por \citeonline{navetsimonotlion}, que está relacionado com a funcionalidade de piloto automático \textit{(cruise control)} presente em alguns modelos de veículos. Para que esse sistema funcione perfeitamente, segundo os autores, é necessário a troca de dados de vários sensores que podem pertencer a categorias funcionais distintas. A função de piloto automático tem a responsabilidade central de processar esses dados vindos de diversas origens e enviar as respostas a vários outros dispositivos de saída para cumprir seu objetivo.

Desde a década de 70, como aponta \citeonline{navetsimonotlion}, houve um aumento muito grande no número de sistemas que passaram a substituir os que eram puramente mecânicos ou hidráulicos. O desempenho e confiabilidade desses componentes de hardware que passaram a integrar o automóvel permite a execução de funções complexas, aumentando o conforto e a segurança dos ocupantes do veículo. A arquitetura de hardware de um automóvel não é composta somente de sensores, atuadores, controladores e links de comunicação que permite a interconexão dos componentes, mas também de dispositivos conhecidos como \textit{ECUs}.

\subsection{\textbf{\textit{ECU (Eletronic Control Unit)}}}
Um veículo possui alguns controladores eletrônicos, que para \citeonline{smith}, são chamados de dispositivos informatizados que passam por diversos nomes diferentes, como Unidade de Controle Eletrônico \textit{(Eletronic Control Unit - ECU)}, Unidade de Controle do Motor \textit{(Engine Control Unit – ECU)}, Unidade de Controle de Transmissão \textit{(Transmission Control Unit – TCU)} ou ainda Módulo de Controle de Transmissão \textit{(Transmission Control Module – TCM)}. O autor ainda destaca que esses termos na teoria podem ter significados específicos em uma determinada configuração, mas na prática acaba utilizando-se o termo \textit{ECU} que é comum a eles. Isso porque independente do tipo de controlador eletrônico, eles acabam executando as mesmas funções, ou funções extremamente semelhantes. \citeonline{navetsimonotlion} afirmam que um dos principais propósitos dos sistemas eletrônicos é auxiliar o condutor a controlar o veículo. Segundo eles, a \textit{ECU} autônoma é um subsistema composto por um microcontrolador e um conjunto de sensores e atuadores associados. Os autores também trazem a ideia de que as \textit{ECUs} podem aprimorar a ação dos atuadores, indo muito além da capacidade humana sobre eles. A Figura \ref{Fig:ecu} apresenta a foto de uma \textit{ECU}.

\begin{figure}[!ht]
\centering
\caption{Foto de uma \textit{ECU}.} 
{\includegraphics[scale=1.3]{imagens/ecu1.jpg}}\\
\makebox[\width]{Fonte: imagem extraída do artigo de \citeonline{carrosinfoco} do site Carros Infoco} \label{Fig:ecu}
\end{figure}

Em outras palavras, a \textit{ECU}, de forma geral, é um dispositivo capaz de processar as informações recebidas pelos sensores dos veículos, e gerar uma resposta para a ação dos atuadores, conforme está representado na Figura \ref{Fig:relacao_ecu_sensor_atuador}. Como a \textit{ECU} faz parte de um sistema embarcado, sua memória e capacidade de processamento são limitadas assim como qualquer dispositivo embarcado, conforme abordado na seção \ref{secaosistemasembarcados}. Entretanto, as \textit{ECUs} são destinadas a executar determinadas tarefas em específico, de modo que o desempenho do sistema como um todo não seja afetado pela limitação de hardware do dispositivo.

\begin{figure}[!ht]
\centering
\caption{Diagrama da comunicação entre \textit{ECU}, sensor e atuador.} 
{\includegraphics[scale=.34]{imagens/ecuSensorAtuador.png}}\\
\makebox[\width]{Fonte: baseado em \citeonline{navetsimonotlion}} \label{Fig:relacao_ecu_sensor_atuador}
\end{figure}

Segundo Smith, a comunicação da \textit{ECU} com os diversos componentes acontece de forma simplificada. Para esta comunicação ser simples e eficiente, os automóveis utilizam alguns protocolos para tratar da comunicação da rede interna do veículo. Estes protocolos serão explorados na sequência.

\subsection{\textbf{Protocolos automotivos}}
Apesar dos protocolos permitirem a comunicação entre os dispositivos, \citeonline{smith} reforça que caso o veículo tenha sido fabricado antes do ano de 2000, há possibilidade de ele não possuir nenhum protocolo implementado. Para ele, os protocolos presentes nos barramentos são responsáveis por gerenciar a transmissão de pacotes de dados pela rede automotiva. Diversos sensores e dispositivos conectados nesta rede se comunicam com a finalidade de administrar o comportamento do veículo. Segundo o autor, toda a parte da comunicação crítica ocorrem nos barramentos de alta velocidade, enquanto a comunicação não crítica ocorre nos barramentos de média e baixa velocidade. Subentende-se que a comunicação crítica está relacionada a todos os controles que garantem a segurança e integridade dos ocupantes do automóvel. 

\citeonline{navetsimonotlion} afirmam que o papel central das redes está em manter os sistemas embarcados em estado de segurança, uma vez que a maioria das funções críticas estão distribuídas e necessitam da comunicação contínua entre si. Ele ainda aponta que a diversificação das redes utilizadas em todo o automóvel é justificada pela crescente necessidade da largura de banda, desempenho e outros requisitos de confiabilidade.

Cada fabricante decide qual barramento ou protocolo serão utilizados na arquitetura do veículo; entretanto, existe um protocolo comum em todos os veículos: o protocolo \textit{CAN} \cite{smith}. Em 1994, a Sociedade de Engenheiros Automotivos \textit{(Society for Automotive Engineers – SAE)}, citada por \citeonline{navetsimonotlion}, definiu uma classificação para protocolos de comunicação automotiva baseando-se na velocidade de transmissão dos dados e nas funções que seriam distribuídas pela rede (Tabela \ref{tab:tipos_redes_transmissao}). As redes de classe A possuem uma taxa de dados inferior a 10 kbps e são utilizadas para transmissão de dados de controle com tecnologias de baixo custo e estão integradas na categoria \textit{Body}. As redes classe B operam em uma velocidade de 10 a 125 kbps e se dedicam para suportar a troca de dados entre \textit{ECUs}, para reduzir o número de sensores compartilhando informações. Aplicações que precisam de comunicação em tempo real requerem redes de classe C (com velocidade de transmissão de 125 kbps a 1Mbps) ou redes de classe D (operando com velocidades superiores a 1Mbps). As redes classe C integram a comunicação das categorias \textit{Power Train} e \textit{Chassis}. Já as redes de classe D são dedicadas a dados multimídia e aplicações críticas de segurança que requeiram previsibilidade e tolerância de falhas.

\begin{table}[htb!]
\centering
\caption{Tipos de redes e suas respectivas velocidades de transmissão de acordo com a \textit{SAE}.}
\label{tab:tipos_redes_transmissao}
\begin{tabular}{cc}
\hline
Tipos de redes & \multicolumn{1}{c}{Velocidade de transmissão} \\ \hline
Classe A              & até 10 kbps                             \\
Classe B              & de 10 a 125 kbps                          \\
Classe C              & de 125 kbps a 1Mbps                       \\
Classe D              & acima de 1Mbps                            \\ \hline
\end{tabular}
\end{table}

De acordo com \citeonline{navetsimonotlion}, é comum a inclusão dos quatro tipos de redes em barramentos diferentes interligadas por \textit{gateways} na arquitetura eletrônica dos veículos atuais, conforme representado na Figura \ref{Fig:redes_gateway}. Eles ainda complementam que será possível futuramente a inclusão de um barramento dedicado aos sistemas de segurança dos ocupantes. De todos os protocolos existentes e citados pelos autores, serão explorados dois, o protocolo \textit{CAN} e o SAE J1850.

\begin{figure}[!ht]
\centering
\caption{Representação de redes distintas interligadas.} 
{\includegraphics[scale=.35]{imagens/redesGateway.png}}\\
\makebox[\width]{Fonte: baseado em \citeonline{navetsimonotlion}} \label{Fig:redes_gateway}
\end{figure}

\subsubsection{\underline{\textbf{Protocolo \textit{CAN} \textit{(Controller Area Network)}}}}
Por volta de 1980, foi desenvolvido pela Bosch o protocolo conhecido como \textit{CAN} \textit{(Controller Area Network)}, ou simplesmente protocolo de controle de área de rede. Este protocolo foi integrado pela primeira vez em carros de produção da Mercedes na década de 90 \cite{navetsimonotlion}. \citeonline{smith} afirma que o \textit{CAN} é um protocolo simples utilizado na indústria automobilística. Este protocolo permite a comunicação de \textit{ECUs} e sistemas embarcados que estão presentes nos veículos modernos. \citeonline{navetsimonotlion} ainda complementam que as redes que utilizam esse protocolo tornaram-se as mais utilizadas nos sistemas automotivos. Por possuírem baixo custo, serem robustas e terem atrasos baixos de comunicação, o protocolo \textit{CAN} é um padrão utilizado na Europa para transmissão de dados de aplicações automotivas.

Atualmente, para controlar os sistemas da categoria \textit{Power Train} utiliza-se a \textit{CAN} como uma rede \textit{SAE} de classe C, operando de 250 a 500 kbps. Entretanto, ela também pode operar nos sistemas de categoria \textit{Body}, com uma taxa de 125 kbps. Segundo os estudos de \citeonline{smith}, o protocolo \textit{CAN} possui dois tipos de pacotes que são chamados de \textit{standard} (padrão) e \textit{extended} (extendido).

Os pacotes \textit{standard} possuem quatro elementos, que são o ID de arbitragem, extensão de ID, código do tamanho dos dados e os própios dados. Quando um dispositivo tenta se comunicar é enviado ID de arbitragem, que é uma mensagem de broadcast que identifica o ID deste dispositivo. Quando dois pacotes são transmitidos ao mesmo tempo no barramento, aquele com ID de arbitragem menor vence. \citeonline{navetsimonotlion} apontam esta técnica como sendo um método utilizado por este protocolo para evitar colisões ao acessar o barramento. Voltando aos estudos de \citeonline{smith}, ele afirma que o bit que define a extensão de ID tem o valor 0 para pacotes \textit{standard}. O código do tamanho dos dados define o tamanho dos dados, que pode variar de 0 a 8 bytes. O tamanho máximo dos dados transportados por este pacote pode ter no máximo 8 bytes.

Já os pacotes \textit{extended}, segundo o autor, são semelhantes ao \textit{standard}, com exceção de possuir espaço maior para armazenar IDs mais longos. Estes pacotes foram desenvolvidos para caberem dentro do formato \textit{standard} para manter a compatibilidade com as versões anteriores. Desta forma, caso algum sensor tenha suporte somente para pacotes \textit{standard}, ele não será invalidado caso sejam transmitidos pacotes \textit{extended} na mesma rede. Dentre outras particularidades deste pacote, \citeonline{smith} ainda reforça que existem outros protocolos adicionais específicos de alguns fabricantes (protocolo ISO-TP, \textit{CANopen}, GMLAN) que seguem o padrão \textit{CAN}, da mesma forma que o pacote \textit{CAN} \textit{extended} (Figura \ref{Fig:implementacoes_can}).

\begin{figure}[!ht]
\centering
\caption{Representação de outros protocolos implementados no padrão \textit{CAN}.} 
{\includegraphics[scale=.25]{imagens/protocoloCanEVariacoes.png}}\\
\makebox[\width]{Fonte: baseado em \citeonline{smith}} \label{Fig:implementacoes_can}
\end{figure}

Segundo \citeonline{navetsimonotlion}, o protocolo \textit{CAN} define apenas a camada física e a camada de link dos dados. Para tratar a padronização de inicialização, implementar a segmentação de dados ou enviar mensagens periódicas, foram propostos diversos protocolos de nível superior.

\subsubsection{\underline{\textbf{Protocolo SAE J1850}}}
O protocolo SAE J1850 foi implementado originalmente por volta de 1994, segundo \citeonline{smith}, e ainda pode ser encontrado nos veículos atuais. Comparado com o protocolo \textit{CAN}, este é mais lento, entretanto, seu custo de implantação é mais barato. \citeonline{navetsimonotlion} menciona que são definidas duas variações para este protocolo. \citeonline{smith} as define como modulação de largura de pulso \textit{(pulse width modulation – PWM)} e largura de pulso variável \textit{(variable pulse width – VPW)}, conforme representado pela Figura \ref{Fig:implementacoes_saej1850}. O \textit{PWM} trabalha com uma velocidade de 41.6 Kbps, enquanto o \textit{VPW} trabalha com 10.4 Kbps.

\begin{figure}[!ht]
\centering
\caption{Variações do protocolo SAE J1850.} 
{\includegraphics[scale=.25]{imagens/protocoloSaeJ1850.png}}\\
\makebox[\width]{Fonte: baseado em \citeonline{smith, navetsimonotlion}} \label{Fig:implementacoes_saej1850}
\end{figure}

\citeonline{navetsimonotlion} afirmam que para tratar dos controles dos sistemas da categoria \textit{Body} ou de diagnósticos, os Estados Unidos adotaram este protocolo SAE J1850 de classe B pois as comunicações não possuíam requisitos de transmissão em tempo real.

Analisando os protocolos apresentados, observa-se que para determinada aplicação é utilizado um protocolo específico.  E cada protocolo tem uma classificação quanto à sua velocidade de transmissão pelo \textit{SAE}. Existem outros protocolos que estão presentes na rede automotiva, como o \textit{Keyword Protocol} e ISO 9141-2, explorados por \citeonline{smith}, os protocolos IDB-1394, \textit{VAN (Vehicle Area Network)} e \textit{TTCAN}, explorados por \citeonline{navetsimonotlion}, ou ainda os protocolos \textit{LIN (Local Interconnect Protocol)}, \textit{MOST (Media Oriented Systems Transport)} e o \textit{FlexRay}, todos explorados na literatura de ambos.

Observando-se o comportamento da rede automotiva, nota-se a presença de vários protocolos responsáveis pela comunicação de diversos dispositivos que são destinados a realizar uma determinada função específica no veículo. Entretanto, também existe a necessidade de monitorar essa rede para realizar eventuais diagnósticos neste sistema, e o nome dado para esta função se chama diagnóstico de bordo \textit{(OnBoard Diagnostics – OBD)} \cite{navetsimonotlion}.

\subsection{\textbf{Conector \textit{OBD-II (OnBoard Diagnostics)}}}\label{subsecaoobd}
Também conhecido como conector de diagnóstico de conexão \textit{(Diagnostic Link Connector – DLC)}, o conector \textit{OBD-II} está integrado em grande parte dos veículos, e permite a comunicação com a rede interna automotiva conforme exemplificado por \citeonline{smith}. De acordo com \citeonline{navetsimonotlion}, a introdução de sistemas informáticos capazes de memorizar grandes quantidades de informação de um automóvel possibilitou o diagnóstico de bordo, que se refere ao autodiagnóstico e a facilidade de emissão de relatórios. Para \citeonline{smith}, o conector \textit{OBD-II} é utilizado principalmente pela mecânica para analisar e solucionar eventuais problemas com o automóvel. A Figura \ref{Fig:obd2} mostra a localização do conector \textit{OBD-II} em um automóvel.

\begin{figure}[!ht]
\centering
\caption{Foto do conector \textit{OBD-II}.} 
{\includegraphics[scale=.19]{imagens/obdII.png}}\\
\makebox[\width]{Fonte: foto tirada pelo autor} \label{Fig:obd2}
\end{figure}

Ao apresenta uma falha, o sistema guarda informações relacionadas a ela e aciona uma luz de aviso do motor no painel do motorista, conhecida como luz indicadora de mau funcionamento (\textit{malfunction indicator lamp – MIL}, Figura \ref{Fig:luz_mil}). Embora o diagnóstico tenha se limitado a essa luz indicadora, conforme \citeonline{navetsimonotlion} apresentam, a comunicação dos sistemas \textit{OBD} recentes são padronizados, como por exemplo a padronização de dados monitorados, codificação padronizada e relatórios de uma lista de falhas específicas, conhecidas como códigos de diagnósticos de problemas \textit{(Diagnostic Trouble Codes – DTC)}. De acordo com \citeonline{smith}, as verificações de rotina são tratadas pela \textit{ECU} primária do veículo, o qual ele se refere de módulo de controle do \textit{powertrain} \textit{(PCM)}. Para controlar a emissão de gases de escape ao longo da vida útil de um veículo, segundo \citeonline{navetsimonotlion}, foi necessário uma padronização na especificação \textit{OBD-II}, que passou a ser obrigatória para todos os carros comercializados nos Estados Unidos a partir de 1996. Este padrão definia com precisão diversos aspectos relacionados ao diagnóstico, avaliação e monitoramento do veículo.

\begin{figure}[!ht]
\centering
\caption{Foto da luz \textit{MIL} do painel.} 
{\includegraphics[scale=.15]{imagens/luzMil.jpg}}\\
\makebox[\width]{Fonte: imagem extraída do site https://goo.gl/CiSaec} \label{Fig:luz_mil}
\end{figure}

Voltando à parte de diagnósticos, com base nos estudos de \citeonline{smith}, todos os códigos de falha, como os \textit{DTCs} são armazenados no \textit{PCM}. Ele ainda complementa dizendo que os \textit{DTCs} são armazenados em locais diferentes. Enquanto os \textit{DTCs} baseados em memória ficam armazenadas na memória RAM e apagados quando a energia da bateria acaba, os \textit{DTCs} mais sérios que estão relacionados com falhas permanentes ficam armazenados em locais onde a persistência de dados é maior e consequentemente sobreviverão à uma queda de energia.

A Figura \ref{Fig:rede_veicular} representa de maneira simplificada como ocorre a comunicação entre os dispositivos na rede interna automotiva.

\begin{figure}[!ht]
\centering
\caption{Diagrama representando a arquitetura da rede veicular.} 
{\includegraphics[scale=.32]{imagens/arquiteturaRedeVeicular.png}}\\
\makebox[\width]{Fonte: baseado em \citeonline{navetsimonotlion} e \citeonline{smith}} \label{Fig:rede_veicular}
\end{figure}

\section{ELM327}
A lei atualmente obriga, segundo a \citeonline{elmeletronics}, que todos os automóveis produzidos forneçam uma interface para a conexão de dispositivos de teste e diagnóstico. Esta interface apresentada na subseção \ref{subsecaoobd} como \textit{OBD-II}, segue as mesmas especificações dos protocolos da rede interna automotiva, o que permite a interação com ela somente se utilizar os mesmos padrões que foram estabelecidos por esta rede. Entretanto, a \citeonline{elmeletronics} ressalta que tais padrões não são compatíveis com computadores ou outros dispositivos inteligíveis, como um smartphone, por exemplo. Para solucionar este problema e permitir a compatibilidade, o ELM327 foi desenvolvido para servir de intermediário entre a porta \textit{OBD-II} - presente nos automóveis - com a interface serial padrão (RS232) presente nos computadores. A \citeonline{elmeletronics} ainda afirma que este dispositivo é capaz de identificar e interpretar nove protocolos automotivos mais o padrão J1939 utilizado por ônibus e caminhões. Desta maneira, com o ELM327 é possível utilizar um computador para se comunicar com a rede interna de um automóvel.

O ELM327 se conecta ao computador por meio da interface serial, que segundo a \citeonline{elmeletronics}, esta interface pode ser virtualizada com adaptadores USB, ou por meio de dispositivos com suporte \textit{bluetooth}. O fabricante reforça que o meio que o dispositivo usa para se conectar ao computador não tem muita relevância, pois para se comunicar com o veículo através da rede, basta apenas uma aplicação que faça o envio ou recebimento dos dados. Entretanto, para garantir a comunicação, alguns ajustes devem ser configurados, como a configuração da porta e a taxa de dados correta, conforme abordados no manual. A Figura \ref{Fig:elm327} mostra como é um dispositivo ELM327 \textit{Bluetooth}, e a Figura \ref{Fig:rede_veicular_elm327} expressa como este dispositivo se integra com o automóvel.

\begin{figure}[!ht]
\centering
\caption{Foto do ELM327 \textit{Bluetooth}.} 
{\includegraphics[scale=.16]{imagens/elm327-min.png}}\\
\makebox[\width]{Fonte: foto tirada pelo autor} \label{Fig:elm327}
\end{figure}

\begin{figure}[!ht]
\centering
\caption{Diagrama representando a interação do ELM327 com a rede automotiva.} 
{\includegraphics[scale=.35]{imagens/arquiteturaRedeVeicularELM327.png}}\\
\makebox[\width]{Fonte: baseado em \citeonline{elmeletronics}} \label{Fig:rede_veicular_elm327}
\end{figure}

Existem dois tipos de comandos que podem ser enviados. Os comandos que se destinam ao próprio dispositivo, e os comandos destinados ao veículo. Segundo a \citeonline{elmeletronics}, os comandos destinados ao dispositivo começam com os caracteres ‘AT’, enquanto os destinados à rede veicular contêm dígitos hexadecimais. Ele ainda ressalta que o ELM327 apenas converte os protocolos, não realizando nenhum tipo de validação dos dados transmitidos. Entretanto, este dispositivo garante a entrega dos dados nas extremidades, ou seja, garante que os dados sejam transmitidos tanto para o computador quanto para o veículo.

\subsection{\textbf{Comandos AT}}
Como visto anteriormente, os comandos ‘AT’ são destinados ao próprio dispositivo ELM327. Existem vários parâmetros dentro dele que podem ser ajustados para modificar seu comportamento, segundo o fabricante. Esses comandos são semelhantes aos utilizados pelos modems para a configuração interna \cite{elmeletronics}.

\subsection{\textbf{Comandos \textit{OBD}}}
Quando um comando é enviado sem as iniciais ‘AT’, é entendido como comando \textit{OBD} para o veículo. Desta forma, o dispositivo apenas verifica se o comando enviado é um algarismo hexadecimal ou uma sequencia aleatória de caracteres, e logo após transmite o dado ao veículo. Para a \citeonline{elmeletronics}, estes dados são empacotados e enviados ao automóvel. Os comandos enviados ao veículo necessitam geralmente de mais quatro bytes adicionais: três bytes de cabeçalho e um byte de verificação de erro, que devem ser incluídos junto aos dados que serão enviados. Contudo, o ELM327 adiciona estes bytes extras ao comando, abstraindo esta necessidade do usuário.

O comprimento dos comandos \textit{OBD} pode variar de um a sete bytes, sendo este último o limite máximo aceito pelo dispositivo. Ao empacotar e enviar o comando pela interface \textit{OBD}, o dispositivo fica monitorando o barramento veicular para obter a resposta ao comando. Obtida a resposta, ela será encaminhada para a porta serial ao usuário.

\subsubsection{\underline{\textbf{Comunicação com o veículo - padrões de solicitação}}}
De acordo com a \citeonline{elmeletronics}, o formato das solicitações que são enviadas ao veículo devem seguir um padrão definido. O primeiro byte enviado é denominado ‘modo’ \textit{(mode)}. Este byte informa que tipo de dados está sendo solicitado. O segundo byte é conhecido como ID de parâmetro, ou simplesmente número PID \textit{(parameter identification)}. Este especifica a informação real que é necessária, como acessar uma informação de um determinado dispositivo do automóvel, por exemplo. Os modos e PIDs são descritos em detalhes pelos padrões SAE J1979 ou ISO 15031-5, além de poderem ser definidos também pelos fabricantes dos automóveis. Entretanto, como menciona a \citeonline{elmeletronics}, é comum um veículo não suportar todos os ‘modos’ e PIDs descritos pelo padrão. As mensagens de resposta, que geralmente são enviadas pela \textit{ECU}, são retornadas em conjunto de números hexadecimais, havendo a necessidade de realizar algumas conversões para ter acesso aos valores que foram retornados. Uma segunda conversão é necessária de acordo com o PID que foi solicitado (a conversão do valor varia de acordo com o PID).

\section{\textit{RASPBERRY PI}}
Segundo \citeonline{oliveira}, o \textit{Raspberry Pi} veio ao mercado com a proposta de ser um equipamento barato e dar suporte ao processo educacional das crianças. Segundo uma entrevista de Torvalds\nocite{torvalds} ao site \textit{BBC News} em 2012, o projeto \textit{Raspberry Pi} é algo importante, pois por ser de baixo custo, permite a exploração da informática sem a preocupação com possíveis danos ao hardware.

\citeonline{richardsonwallace} afirmam que é possível realizar diversas atividades com o \textit{Raspberry Pi}, como utilizá-lo para computação de maneira geral, aprender programação ou ainda integrar o minicomputador a projetos eletrônicos. Segundo os autores, um dos fatores que o diferenciam de um computador convencional além do seu tamanho e preço, é sua facilidade de integração com projetos eletrônicos. Apesar de parecer semelhante aos microcontroladores, as plataformas \textit{System on a Chip (SoC)} possuem mais características em comum com um computador do que com um microcontrolador qualquer. A Figura \ref{Fig:raspberry_pi} mostra a aparência física do \textit{Raspberry Pi}.

\begin{figure}[!ht]
\centering
\caption{Foto do \textit{Raspberry Pi} 3.} 
{\includegraphics[scale=.20]{imagens/raspberry-min.png}}\\
\makebox[\width]{Fonte: foto tirada pelo autor} \label{Fig:raspberry_pi}
\end{figure}

O \textit{Raspberry Pi}, segundo \citeonline{oliveira}, é um computador montado em apenas uma única placa, o qual se classifica como \textit{Single-Board Computer (SBC)}. Ele ainda afirma que nesta placa, são integrados o processador, memória, portas de I/O e outros componentes que são necessários para seu funcionamento. A arquitetura presente em seu processador é um ARM, semelhantes aos processadores utilizados em celulares, tablets ou sistemas embarcados. \citeonline{richardsonwallace} afirmam que este processador é o mesmo que os encontrados no iPhone 3G e no \textit{Kindle} 2, o que permite ter uma referência sobre seu poder de processamento. De acordo com os autores, os chips ARM possuem diferentes núcleos configurados para fornecer capacidades diferentes de processamento. Seu sistema operacional padrão é uma distribuição Linux conhecida como \textit{Raspbian}, entretanto, os autores ressaltam que o usuário final não é limitado a utilizar somente este S.O. no dispositivo, ficando livre para explorar outras distribuições.

\section{COMPUTAÇÃO EM NUVEM}
De acordo com a definição da \citeonline{microsoft}, computação em nuvem é a disponibilização de diversos serviços de computação - como servidores, armazenamento, banco de dados, sistemas, entre outros serviços de TI - através da internet. A \textit{Amazon Web Service} \nocite{amazoncloudcomputing} ainda complementa que a plataforma que provê estes serviços é responsável pela manutenção de todo equipamento necessário para a disponibilização dos serviços. Tanto a \textit{Amazon Web Services (AWS)} quanto a Microsoft \textit{Azure} são plataformas conhecidas que proveem serviços de computação em nuvem.

Ambas afirmam que existem vários benefícios na utilização destes serviços. Dentre eles, uma das vantagens é relacionada ao custo, pois dispensa o gasto com qualquer tipo de equipamento para manter um datacenter local, uma vez que é utilizado uma infraestrutura já pronta e montada por estas prestadoras de serviço, que garantem a disponibilidade dos recursos através da internet \cite{amazoncloudcomputing, microsoft}.

Estes serviços são classificados em três tipos: Infraestrutura como Serviço (IaaS), Plataforma como Serviço (PaaS) e Software como Serviço (SaaS). A categoria de IaaS, segundo a definição da \citeonline{microsoft}, é a mais básica e provém toda a infraestrutura de TI, como servidores, máquinas virtuais, armazenamento, redes e sistemas operacionais. A categoria PaaS são serviços de computação que fornecem ambientes sob demanda para desenvolvimento, teste, fornecimento e gerenciamento de aplicativos de software. O SaaS fornece aplicativos de software pela nuvem sob demanda, normalmente baseadas em assinaturas.

\section{INTERNET DAS COISAS \textit{(INTERNET OF THINGS - IOT)}}
A ideia de \textit{IoT} se refere aos objetos do mundo físico gerando informações de forma autônoma para os computadores \cite{ashton}. Ele reforça que todas as informações presentes na internet foram geradas a partir de um ser humano. Uma questão pontuada por ele é referida à limitação das pessoas com relação ao tempo, atenção e precisão. Essa limitação pode resultar em ineficiência ao capturar dados sobre o mundo real. Dito isso, o autor conclui afirmando que é necessário capacitar os computadores através de seus próprios meios de coleta de informações para observarem e compreenderem o mundo físico sem a limitação da entrada de dados através de uma pessoa.

O conceito de Internet das Coisas conforme definido pelo projeto \citeonline{casagras} na introdução e idealizado por \citeonline{ashton} abordam a interconectividade e troca de informações entre os objetos do mundo físico com o mundo virtual (Figura \ref{Fig:representacao_iot}). \citeonline{dias} afirma que existem infinitas possibilidades de negócio e aplicações envolvendo o conceito de internet das coisas. Ela ainda menciona alguns nichos destes negócios, como os voltados para os bens de consumo, distribuição de energia, casas inteligentes, indústria e manufatura e transporte inteligente. Para este último nicho, a autora ainda traz algumas aplicações possíveis de serem exploradas, como notificação das condições de tráfego, controle inteligente de rotas, coordenação das rodovias e monitoramento remoto de veículos.

\begin{figure}[!ht]
\centering
\caption{Representação da interação das 'coisas' dentro do conceito de \textit{IoT}.} 
{\includegraphics[scale=.40]{imagens/diagramaIOT.png}}\\
\makebox[\width]{Fonte: baseado nas obras de \citeonline{ashton}, \citeonline{casagras} e \citeonline{dias}} \label{Fig:representacao_iot}
\end{figure}

De acordo com o que foi apresentado até o momento, observam-se dois pontos importantes: primeiro, a cada evolução de um automóvel, nota-se que ele está se tornando mais informatizado e conectado; segundo, a tendência da \textit{IoT} é permitir que cada vez mais as coisas se comuniquem e troquem informações entre si, com mínima intervenção humana.

\chapter{METODOLOGIA}\label{CAP5}

O processo de desenvolvimento do trabalho foi dividido em 3 etapas. A primeira etapa consistiu no desenvolvimento do software responsável por fazer a interação com a rede interna automotiva. A segunda etapa foi caracterizada pela configuração do \textit{Raspberry Pi} e instalação do software no dispositivo e a terceira etapa se refere à integração da aplicação com um serviço de computação em nuvem.

\section{DESENVOLVIMENTO DO SOFTWARE DE LEITURA}
O objetivo principal da concepção do software nesta fase inicial é possibilitar ao computador interagir com a rede interna veicular, enviando e recebendo dados através da interface \textit{OBD-II}. Para permitir a tradução de protocolos e tornar a comunicação simplificada, foi adquirido o dispositivo ELM327 com transmissão \textit{bluetooth}, responsável por tratar das conversões dos protocolos automotivos presentes no conector \textit{OBD-II} para uma interface serial padrão, estabelecendo uma comunicação com o adaptador \textit{bluetooth} do computador.

A construção do software foi feita utilizando a linguagem de programação Java na versão 8, junto com as seguintes tecnologias da linguagem: API \textit{BlueCove}, API obd-java-api e o \textit{framework} JavaFx. O \textit{BlueCove} é uma biblioteca em Java que segue a implementação JSR-82, permitindo que uma aplicação acesse o adaptador \textit{bluetooth} local para se comunicar com outros dispositivos \cite{bluecove}. A API obd-java-api, que foi encontrada no repositório de Pires (https://github.com/pires/obd-java-api), é uma biblioteca que implementa várias classes e métodos que facilita a comunicação com o adaptador ELM327. O \textit{framework} JavaFx fornece um conjunto de pacotes que facilitam a criação de uma interface gráfica para uma aplicação \cite{pawlan}. A Figura \ref{Fig:tela_leitura_javafx} representa o protótipo do software desenvolvido utilizando o JavaFx.

\begin{figure}[!ht]
\centering
\caption{Imagem da tela de leitura prototipada.} 
{\includegraphics[scale=.75]{imagens/telaLeituraJavaFx.PNG}}\\
\makebox[\width]{Fonte: produzido pelo autor} \label{Fig:tela_leitura_javafx}
\end{figure}

\subsection{ARQUITETURA DO SOFTWARE}
O projeto foi desenvolvido utilizando o paradigma de programação orientada à objeto e o padrão \textit{Model-View-Controller (MVC)}, e devido a isso, foi necessário criar cinco pacotes para organizar as classes do código fonte. Os nomes dos pacotes criados foram: \textit{app}, \textit{bluetooth}, \textit{controller}, \textit{scanner}, e \textit{view}.

No pacote \textit{app} contém a classe \textit{AppStart} que é responsável por iniciar a aplicação e carregar a interface de usuário \textit{ScreenMonitor} contida no pacote \textit{view}. O pacote \textit{bluetooth} contém duas classes, uma chamada \textit{DiscoveryDevices} – responsável pela descoberta de dispositivos \textit{bluetooth} próximos – e outra chamada \textit{BluetoothConnection}, responsável por obter uma conexão \textit{bluetooth}. Dentro do pacote \textit{scanner} existem duas classes, uma com o nome \textit{ConnectToDevice}, que é responsável por estabelecer uma conexão com o dispositivo ELM327, e outra com o nome ELM327, representando o próprio dispositivo. No pacote \textit{view} contém o arquivo \textit{ScreenMonitor} no formato FXML que é responsável por renderizar a interface de usuário. E por fim, o pacote \textit{controller}, que contém a classe \textit{ScreenMonitorController}, responsável por receber as entradas do usuário na interface e mapear as ações a serem tomadas pelo software. Na Figura \ref{Fig:diagrama_classe} é possível observar a organização e estruturação das classes no projeto.

\begin{figure}[!ht]
\centering
\caption{Diagrama de classes exibindo a estrutura de pacotes do projeto com as respectivas classes.} 
{\includegraphics[scale=.44]{imagens/estruturacaoPacotes.PNG}}\\
\makebox[\width]{Fonte: produzido pelo autor} \label{Fig:diagrama_classe}
\end{figure}

Analisando o pacote \textit{bluetooth}, a classe \textit{BluetoothConnection} traz um método estático chamado \textit{getConnectionBluetooth} (Figura \ref{Fig:get_connection_bluetooth}) que espera como parâmetro uma URL de conexão. Este método que tem a responsabilidade de obter uma conexão \textit{bluetooth} devolvendo um objeto do tipo \textit{StreamConnection}, pertencente à biblioteca \textit{BlueCove}. A Classe \textit{DiscoveryDevices} (Figura \ref{Fig:discovery_devices}) implementa a interface \textit{DiscoveryListener}, também pertencente à biblioteca \textit{BlueCove}, que permite a descoberta de dispositivos e serviços. Esta interface fornece quatro métodos para serem implementados: dois para descobrir dispositivos, que são o \textit{deviceDiscovered} (Figura \ref{Fig:device_discovered}) – método que é invocado quando é encontrado um dispositivo durante uma consulta – e o \textit{inquiryCompleted} (Figura \ref{Fig:inquiry_completed}) – método que é chamado quando uma consulta é concluída, e dois para descobrir serviços, que são o \textit{servicesDiscovered} (Figura \ref{Fig:services_discovered}) – são invocados quando os serviços são encontrados durante uma pesquisa por serviços – e o \textit{serviceSearchCompleted} (Figura \ref{Fig:service_search_completed}) – são chamados quando uma pesquisa de serviço foi concluída ou encerrada devido a um erro \cite{bluecovedoc}. Além destes métodos contidos na assinatura da interface, a classe \textit{DiscoveryDevices} também traz o método \textit{discovery} (Figura \ref{Fig:discovery}), responsável por iniciar a descoberta de dispositivos.

\begin{figure}[!ht]
\centering
\caption{Foto do método \textit{getConnectionBluetooth} da classe \textit{BluetoothConnection}.} 
{\includegraphics[scale=.80]{imagens/pacoteBluetooth-BluetoothConnection.PNG}}\\
\makebox[\width]{Fonte: produzido pelo autor} \label{Fig:get_connection_bluetooth}
\end{figure}

\begin{figure}[!ht]
\centering
\caption{Foto da Classe \textit{DiscoveryDevices} implementando a interface \textit{DiscoveryListener}.} 
{\includegraphics[scale=.94]{imagens/pacoteBluetooth-DiscoveryDevices_DiscoveryListener.PNG}}\\
\makebox[\width]{Fonte: produzido pelo autor} \label{Fig:discovery_devices}
\end{figure}

\begin{figure}[!ht]
\centering
\caption{Foto do método \textit{deviceDiscovered} da classe \textit{DiscoveryDevices}.} 
{\includegraphics[scale=.78]{imagens/pacoteBluetooth-DiscoveryDevices_deviceDiscovered.PNG}}\\
\makebox[\width]{Fonte: produzido pelo autor} \label{Fig:device_discovered}
\end{figure}

\begin{figure}[!ht]
\centering
\caption{Foto do método \textit{inquiryCompleted} da classe \textit{DiscoveryDevices}.} 
{\includegraphics[scale=.80]{imagens/pacoteBluetooth-DiscoveryDevices_inquiryCompleted.PNG}}\\
\makebox[\width]{Fonte: produzido pelo autor} \label{Fig:inquiry_completed}
\end{figure}

\begin{figure}[!ht]
\centering
\caption{Foto do método \textit{servicesDiscovered} da classe \textit{DiscoveryDevices}.} 
{\includegraphics[scale=.78]{imagens/pacoteBluetooth-DiscoveryDevices_servicesDiscovered.PNG}}\\
\makebox[\width]{Fonte: produzido pelo autor} \label{Fig:services_discovered}
\end{figure}

\begin{figure}[!ht]
\centering
\caption{Foto do método \textit{serviceSearchCompleted} da classe \textit{DiscoveryDevices}.} 
{\includegraphics[scale=.78]{imagens/pacoteBluetooth-DiscoveryDevices_serviceSearchCompleted.PNG}}\\
\makebox[\width]{Fonte: produzido pelo autor} \label{Fig:service_search_completed}
\end{figure}

\begin{figure}[!ht]
\centering
\caption{Foto do método \textit{discovery} da classe \textit{DiscoveryDevices}.} 
{\includegraphics[scale=.70]{imagens/pacoteBluetooth-DiscoveryDevices_discovery.PNG}}\\
\makebox[\width]{Fonte: produzido pelo autor} \label{Fig:discovery}
\end{figure}

Observando o pacote \textit{scanner}, existe a classe \textit{ConnectToDevice} (Figura \ref{Fig:connect_to_device}) que espera como parâmetro em seu construtor um objeto do tipo \textit{DiscoveryDevices}. Esta classe também possui o método \textit{connectToDevice} (Figura \ref{Fig:connect_connect_to_device}), que espera como parâmetro um índice referente ao dispositivo que se deseja conectar. Outra classe presente neste pacote é a ELM327 (Figura \ref{Fig:elm327_class}), que recebe como parâmetro em seu construtor dois objetos do tipo \textit{InputStream} e \textit{OutputStream}. Ela também contém alguns métodos, que são: \textit{disconnect} (Figura \ref{Fig:elm327_disconnect}), responsável por fechar a conexão dos objetos \textit{InputStream} e \textit{OutputStream}; \textit{readRpm} (Figura \ref{Fig:elm327_read_rpm}), responsável por efetuar a leitura da rotação por minuto (RPM) do motor, devolvendo uma \textit{string} no formato correto; \textit{readSpeed} (Figura \ref{Fig:elm327_read_speed}), responsável por efetuar a leitura da velocidade atual do automóvel, também retornando uma \textit{string} no formato Km/h; \textit{readFuelPressure} (Figura \ref{Fig:elm327_read_fuel_pressure}), responsável por obter a pressão do combustível em uma \textit{string}, no formato Psi ou Kilopascal (kPa); \textit{readOilTemp} (Figura \ref{Fig:elm327_read_oil_temp}), responsável por ler a temperatura do óleo do motor e devolver uma \textit{string} com o valor em graus \textit{Celsius} ($^{\circ}$C); \textit{readFindFuelType} (Figura \ref{Fig:elm327_read_find_fuel_type}), responsável por encontrar qual tipo de combustível está sendo utilizado no tanque; \textit{readFuelLevel} (Figura \ref{Fig:elm327_read_fuel_level}), responsável por obter a informação referente ao nível de combustível presente no tanque, e por último, o método \textit{clearBuffer} (Figura \ref{Fig:elm327_clear_buffer}), que implementa vários comandos que reiniciam a conexão \textit{OBD}, limpam o eco e o cabeçalho, possibilitando apagar o \textit{buffer} presente no dispositivo ELM327 para realizar novas leituras. Todos os métodos pertencentes à esta classe, exceto o \textit{disconnect}, faz o uso de classes e métodos que pertencem à biblioteca obd-java-api. Desta forma, as classes e métodos contidos nesta biblioteca abstrai a implementação de comandos ‘AT’ e \textit{‘OBD’}, tornando a sua utilização simplificada, bastando apenas chamar o recurso a ser utilizado sem se preocupar com a forma que será implementada a funcionalidade.

\begin{figure}[!ht]
\centering
\caption{Foto da classe \textit{ConnectToDevice}.} 
{\includegraphics[scale=.80]{imagens/pacoteScanner-ConnectToDevice.PNG}}\\
\makebox[\width]{Fonte: produzido pelo autor} \label{Fig:connect_to_device}
\end{figure}

\begin{figure}[!ht]
\centering
\caption{Foto do método \textit{connectToDevice} da classe \textit{ConnectToDevice}.} 
{\includegraphics[scale=.70]{imagens/pacoteScanner-ConnectToDevice_connectToDevice.PNG}}\\
\makebox[\width]{Fonte: produzido pelo autor} \label{Fig:connect_connect_to_device}
\end{figure}

\begin{figure}[!ht]
\centering
\caption{Foto da classe ELM327.} 
{\includegraphics[scale=.70]{imagens/pacoteScanner-ELM327.PNG}}\\
\makebox[\width]{Fonte: produzido pelo autor} \label{Fig:elm327_class}
\end{figure}

\begin{figure}[!ht]
\centering
\caption{Foto do método \textit{disconnect} da classe ELM327.} 
{\includegraphics[scale=.70]{imagens/pacoteScanner-ELM327_disconnect.PNG}}\\
\makebox[\width]{Fonte: produzido pelo autor} \label{Fig:elm327_disconnect}
\end{figure}

\begin{figure}[!ht]
\centering
\caption{Foto do método \textit{readRpm} da classe ELM327.} 
{\includegraphics[scale=.85]{imagens/pacoteScanner-ELM327_readRpm.PNG}}\\
\makebox[\width]{Fonte: produzido pelo autor} \label{Fig:elm327_read_rpm}
\end{figure}

\begin{figure}[!ht]
\centering
\caption{Foto do método \textit{readSpeed} da classe ELM327.} 
{\includegraphics[scale=.85]{imagens/pacoteScanner-ELM327_readSpeed.PNG}}\\
\makebox[\width]{Fonte: produzido pelo autor} \label{Fig:elm327_read_speed}
\end{figure}

\begin{figure}[!ht]
\centering
\caption{Foto do método \textit{readFuelPressure} da classe ELM327.} 
{\includegraphics[scale=.80]{imagens/pacoteScanner-ELM327_readFuelPressure.PNG}}\\
\makebox[\width]{Fonte: produzido pelo autor} \label{Fig:elm327_read_fuel_pressure}
\end{figure}

\begin{figure}[!ht]
\centering
\caption{Foto do método \textit{readOilTemp} da classe ELM327.} 
{\includegraphics[scale=.85]{imagens/pacoteScanner-ELM327_readOilTemp.PNG}}\\
\makebox[\width]{Fonte: produzido pelo autor} \label{Fig:elm327_read_oil_temp}
\end{figure}

\begin{figure}[!ht]
\centering
\caption{Foto do método \textit{readFindFuelType} da classe ELM327.} 
{\includegraphics[scale=.80]{imagens/pacoteScanner-ELM327_readFindFuelType.PNG}}\\
\makebox[\width]{Fonte: produzido pelo autor} \label{Fig:elm327_read_find_fuel_type}
\end{figure}

\begin{figure}[!ht]
\centering
\caption{Foto do método \textit{readFuelLevel} da classe ELM327.} 
{\includegraphics[scale=.70]{imagens/pacoteScanner-ELM327_readFuelLevel.PNG}}\\
\makebox[\width]{Fonte: produzido pelo autor} \label{Fig:elm327_read_fuel_level}
\end{figure}

\begin{figure}[!ht]
\centering
\caption{Foto do método \textit{clearBuffer} da classe ELM327.} 
{\includegraphics[scale=.70]{imagens/pacoteScanner-ELM327_clearBuffer.PNG}}\\
\makebox[\width]{Fonte: produzido pelo autor} \label{Fig:elm327_clear_buffer}
\end{figure}

Os pacotes \textit{view}, e \textit{controller} são implementações da arquitetura \textit{MVC}, que correspondem respectivamente à arquivos relacionados à interface de usuário, e à classes responsáveis por administrar as entradas dos usuários. Segundo \citeonline{medeirosmvc}, a arquitetura \textit{MVC} possui o controlador \textit{(Controller)} que gerencia as entradas dos usuários através das visões \textit{(Views)}, passando os comandos para os modelos \textit{(Models)} que gerencia diversos elementos de dados. Seguindo esta ideia, os pacotes que tem o papel de modelo, segundo esta arquitetura, seria o \textit{scanner} e o \textit{bluetooth} (Figura \ref{Fig:diagrama_mvc}).

\begin{figure}[!ht]
\centering
\caption{Representação da arquitetura \textit{MVC} do projeto.} 
{\includegraphics[scale=.40]{imagens/diagramaMvc.png}}\\
\makebox[\width]{Fonte: produzido pelo autor} \label{Fig:diagrama_mvc}
\end{figure}

\section{CONFIGURAÇÃO DO \textit{RASPBERRY PI} E INSTALAÇÃO DO SOFTWARE}
Primeiramente foi instalado a versão \textit{Jessie} do \textit{Raspbian}, com a atualização de 05 de julho de 2017 disponível em https://downloads.raspberrypi.org/raspbian/images/raspbian-2017-07-05/. O \textit{Raspbian}, conforme menciona \citeonline{long}, é uma distribuição do Linux baseada no Debian. Foi adquirido também, para o \textit{Raspberry} o display de 3.2 polegadas, modelo 3.2inch RPi Display com resolução de 320x240 pixels (Figura \ref{Fig:raspberry_display}). Este display é conectado na porta genérica de entrada e saída do \textit{Raspbery Pi} (porta GPIO - Figura \ref{Fig:raspberry_gpio}). O \textit{driver} de instalação do display está disponível em https://www.waveshare.com/wiki/3.2inch\_RPi\_LCD\_(B). De acordo com as instruções de instalação presentes neste site, os \textit{drivers} não são compatíveis com sistemas instalados pelo NOOBS. O sistema NOOBS, segundo o \citeauthor{raspberrypifoundation}, é um instalador de sistema operacional que contém o \textit{Raspbian}.

\begin{figure}[!ht]
\centering
\caption{Foto do display do \textit{Raspberry Pi}.} 
{\includegraphics[scale=.15]{imagens/displayRaspberry-min.png}}\\
\makebox[\width]{Fonte: produzido pelo autor} \label{Fig:raspberry_display}
\end{figure}

\begin{figure}[!ht]
\centering
\caption{Porta GPIO do \textit{Raspberry Pi}.} 
{\includegraphics[scale=.15]{imagens/raspberryGPIO-min.png}}\\
\makebox[\width]{Fonte: produzido pelo autor} \label{Fig:raspberry_gpio}
\end{figure}

Para suportar a instalação e execução do software, depois de instalado e configurado o sistema operacional e o display, foi instalado o kit de desenvolvimento para Java na versão 8 (JDK 8) que incluía a máquina virtual (JRE) para rodar o projeto. Durante a fase de execução do projeto no ambiente do \textit{Raspberry}, notou-se que houve duas incompatibilidades. A primeira incompatibilidade identificada foi relacionada à biblioteca \textit{BlueCove}, pois ela não fornecia suporte para arquitetura ARM. A solução alternativa para contornar este problema, foi pesquisar a biblioteca compilada para esta arquitetura. A segunda incompatibilidade foi relacionada ao \textit{framework} JavaFx, utilizado na interface gráfica. Para solucionar este problema, foi necessário redesenhar a interface utilizando o \textit{Swing} – conjunto de componentes que permite a criação de interface gráfica –, que são inteiramente implementados na linguagem Java \cite{oracle}.

Foram feitas algumas modificações no projeto e na interface para se adequar à tela do \textit{Raspberry} de 3.2 polegadas. Como havia uma limitação de espaço de exibição no display, apenas as leituras de RPM, velocidade, tipo e pressão do combustível foram implementadas. A Figura \ref{Fig:raspberry_sistema} mostra o software adaptado à tela, sendo executado no \textit{Raspberry Pi}.

\begin{figure}[!ht]
\centering
\caption{Software sendo executado no \textit{Raspberry Pi}.} 
{\includegraphics[scale=.15]{imagens/sistemaRodandoRaspberry-min.png}}\\
\makebox[\width]{Fonte: produzido pelo autor} \label{Fig:raspberry_sistema}
\end{figure}

\section{INTEGRAÇÃO DA APLICAÇÃO COM SERVIÇO DE COMPUTAÇÃO EM NUVEM}
Primeiramente foi feito uma análise da arquitetura necessária que o servidor deveria possuir além de levantar quais serviços e softwares seriam necessários para garantir o funcionamento do sistema na web. Baseando-se nestas informações, foi adquirido um servidor do tipo \textit{Elastic Compute Cloud (EC2)} da Amazon, pertencente à categoria t2.micro, contendo 1Gb de memória RAM, processador Intel Xeon de 2.5GHz e 8Gb de armazenamento do tipo \textit{Elastic Block Store (EBS)}, rodando o sistema operacional Ubuntu \textit{Server} 16.04 LTS. Esta categoria de servidor permite o uso por um ano de forma gratuita. O \textit{EC2}, segundo a \citeonline{amazonec2}, é um \textit{web service} que disponibiliza capacidade computacional segura e redimensionável na nuvem. O \textit{EBS} disponibiliza volumes de armazenamento para uso com instâncias de servidores do tipo \textit{EC2} \nocite{amazonebs}.

\subsection{INSTALAÇÃO E CONFIGURAÇÃO DA BASE DE DADOS NA NUVEM}
Depois de feita a aquisição do servidor, a próxima etapa foi instalar e configurar um serviço de banco de dados. Para esta situação foi escolhido o MongoDB, que é um banco de dados não relacional baseado em documentos totalmente escalável \cite{mongodbwhatis}. Após a instalação, foi criado uma base de dados no MongoDB com o nome \textit{car\_monitor}, e dentro desta base, foi criado uma coleção com o nome \textit{reading\_sensors}. Esta coleção armazenará todos os dados do veículo que foram lidos pelo \textit{Raspberry Pi}. A Figura \ref{Fig:base_dados} representa a estrutura do banco de dados que será responsável por armazenar as informações.

\begin{figure}[!ht]
\centering
\caption{Foto da estrutura do banco de dados.} 
{\includegraphics[scale=1.5]{imagens/baseDadosBanco.png}}\\
\makebox[\width]{Fonte: produzido pelo autor} \label{Fig:base_dados}
\end{figure}

\subsection{INTEGRAÇÃO DO SOFTWARE COM A BASE DE DADOS}
Para a integração do software com a base de dados foi necessário instalar as seguintes dependências do MongoDB na aplicação, que são as bibliotecas: \textit{mongodb-driver}, \textit{mongodb-driver-core} e \textit{bson}, todas na versão 3.5. Para manipular as informações no banco de dados pelo software, foi decidido seguir o modelo \textit{Data Access Object (DAO)}. Segundo a definição de \apudonline{deepak}{medeirosdao}, o padrão \textit{DAO} abstrai e encapsula todos os acessos ao banco de dados, gerenciando a conexão com a base para obter e armazenar as informações. Para implementar este padrão, foi necessário alterar a estrutura do projeto, com a criação de 3 pacotes adicionais: o pacote \textit{dao}, o pacote \textit{database} e o pacote \textit{models}. A Figura \ref{Fig:diagrama_classe_novo} mostra a nova estrutura de pacotes e a Figura \ref{Fig:diagrama_mvc_dao} exibe a representação dos padrões utilizados no projeto baseado nos pacotes implementados.

\begin{figure}[!ht]
\centering
\caption{Diagrama de classes exibindo a nova estrutura de pacotes do projeto com as respectivas classes.} 
{\includegraphics[scale=.44]{imagens/estruturacaoPacotesNovo.png}}\\
\makebox[\width]{Fonte: produzido pelo autor} \label{Fig:diagrama_classe_novo}
\end{figure}

\begin{figure}[!ht]
\centering
\caption{Representação da arquitetura \textit{MVC} com o padrão \textit{DAO} do projeto.} 
{\includegraphics[scale=.38]{imagens/diagramaMvcDao.png}}\\
\makebox[\width]{Fonte: produzido pelo autor} \label{Fig:diagrama_mvc_dao}
\end{figure}

O pacote \textit{models} contém a classe modelo \textit{ELM327ReadSensors} (Figura \ref{Fig:elm327_read_sensors}), que será utilizada para a criação de objetos que armazenarão todas as informações que foram lidas do veículo. Esta classe possui os campos do tipo \textit{string} referentes à estas informações, que são: modeloCarro, chassiCarro, rpm, velocidade, pressaoCombustivel e tipoCombustivel.

\begin{figure}[!ht]
\centering
\caption{Foto da classe \textit{ELM327ReadSensors}.} 
{\includegraphics[scale=.66]{imagens/pacoteModel-ELM327ReadSensors.png}}\\
\makebox[\width]{Fonte: produzido pelo autor} \label{Fig:elm327_read_sensors}
\end{figure}

Analisando o pacote \textit{database}, ele traz a classe \textit{DBConnection} (Figura \ref{Fig:db_connection}) que é responsável por criar a conexão com o banco de dados na nuvem. Esta classe contém o método estático \textit{getConnection} que retorna um objeto do tipo \textit{MongoDatabase}, pertencente à biblioteca \textit{mongodb-driver}.

\begin{figure}[!ht]
\centering
\caption{Foto da classe \textit{DBConnection}.} 
{\includegraphics[scale=.66]{imagens/pacoteDatabase-DBConnection.png}}\\
\makebox[\width]{Fonte: produzido pelo autor} \label{Fig:db_connection}
\end{figure}

No pacote \textit{dao}, existe a classe \textit{ELM327ReadSensorsDAO} (Figura \ref{Fig:elm327_read_sensors_dao}), que implementa o padrão \textit{DAO}. Esta classe abstrai o acesso ao banco de dados, fornecendo um método para inserção chamada \textit{insertDB}, que recebe como parâmetro um objeto do tipo \textit{ELM327ReadSensors}. Este método é responsável por receber o objeto e estrutura-lo em um documento no formato \textit{BSON} para ser inserido no banco de dados. Os dados contidos no MongoDB são documentos \textit{JSON} codificados, conhecidos como \textit{BSON} \cite{mongodbjsonbson}.

\begin{figure}[!ht]
\centering
\caption{Foto da classe \textit{ELM327ReadSensorsDAO}.} 
{\includegraphics[scale=.66]{imagens/pacoteDao-ELM327ReadSensorsDAO.png}}\\
\makebox[\width]{Fonte: produzido pelo autor} \label{Fig:elm327_read_sensors_dao}
\end{figure}

Para permitir o upload dos dados na nuvem utilizando o \textit{Raspberry Pi}, foi levantado a necessidade de utilizar uma rede móvel 4G. Desta forma, seria possível conectar o software com o MongoDB localizado em uma instância da Amazon na nuvem.

\subsection{INSTALAÇÃO E CRIAÇÃO DO \textit{WEB SERVICE}}
Depois de integrado o software com o banco de dados, foi necessário criar um \textit{web service} que seria responsável por realizar as consultas no banco e disponibilizá-las para alguma outra aplicação poder consumir estas informações. Baseado nesta definição, foi necessário a instalação da plataforma Node.js versão 6.11 para a criação do \textit{web service} utilizando a linguagem \textit{JavaScript}. Segundo o \citeonline{w3schools}, o Node.js é um \textit{framework} para servidor de código aberto que utiliza a linguagem \textit{JavaScript} para as aplicações de \textit{web service}, permitindo a manipulação de informações no \textit{backend}.

Foi necessário também a instalação de três pacotes do Node.js para a criação do \textit{web service}. O pacote \textit{express}, responsável pela abstração das rotas, o pacote \textit{body-parser} responsável por efetuar as conversões \textit{JSON} e o pacote \textit{mongoose}, responsável por mapear os objetos do MongoDB. O \textit{web service} foi implementado apenas para requisições HTTP do tipo \textit{GET} na rota \textit{‘/collection’}. Desta maneira, qualquer aplicação que fazer uma solicitação do tipo \textit{GET} para esta rota, será chamado uma função no \textit{web service} (Figura \ref{Fig:get_collection}) que é responsável por realizar a consulta no banco de dados e retornar os valores encontrados em formato de uma lista de objetos \textit{JSON} respondendo à requisição.

\begin{figure}[!ht]
\centering
\caption{Foto do método js responsável por responder a solicitações \textit{GET} na rota \textit{'/collection'}.} 
{\includegraphics[scale=.66]{imagens/GET-collection_webservice.png}}\\
\makebox[\width]{Fonte: produzido pelo autor} \label{Fig:get_collection}
\end{figure}

\subsection{INSTALAÇÃO E CONFIGURAÇÃO DO SERVIDOR WEB E CRIAÇÃO DA PÁGINA}
Após concluída a etapa anterior, seria preciso instalar um servidor web para hospedar a página que iria consumir os dados disponibilizados pelo \textit{web service}. Contudo, foi escolhido o servidor Apache para esta finalidade. As configurações padrões do servidor Apache foram mantidas, pois não haveria necessidade de alterá-las nesta situação.

O desenvolvimento da página web foi feito utilizando as linguagens HTML e \textit{JQuery}, e o \textit{framework} \textit{Bootstrap}. A função desta página é simplesmente consumir as informações disponíveis no \textit{web service}, não possuindo nenhuma outra funcionalidade específica. A página contém um componente de seleção, que permite escolher o tipo de filtro para exibir os resultados. Logo abaixo contém o campo de entrada que permite inserir o valor da informação que se deseja filtrar. A Figura \ref{Fig:input_fitros} apresenta os componentes de filtro mencionados acima.

\begin{figure}[!ht]
\centering
\caption{Foto dos elementos de filtro da página.} 
{\includegraphics[scale=.62]{imagens/paginaweb_inputs.png}}\\
\makebox[\width]{Fonte: produzido pelo autor} \label{Fig:input_fitros}
\end{figure}

Após preenchido o valor, existe um botão chamado monitorar, que ao acionado, faz a requisição \textit{GET} via \textit{AJAX} (Figura \ref{Fig:requisicao_ajax}) para o \textit{web service} e retorna os valores na tabela de acordo com o filtro informado.
Por fim, a Figura \ref{Fig:arquitetura_projeto} mostra toda a integração do sistema com a nuvem, representando as trocas de dados e comunicação entre diferentes serviços.

\begin{figure}[!ht]
\centering
\caption{Foto do método \textit{JQuery} responsável pelo \textit{AJAX}.} 
{\includegraphics[scale=.64]{imagens/requisicaoAjax.png}}\\
\makebox[\width]{Fonte: produzido pelo autor} \label{Fig:requisicao_ajax}
\end{figure}

\begin{figure}[!ht]
\centering
\caption{Diagrama representando a comunicação do sistema com a nuvem.} 
{\includegraphics[scale=.38]{imagens/arquiteturaRedeVeicularELM327Nuvem.png}}\\
\makebox[\width]{Fonte: produzido pelo autor} \label{Fig:arquitetura_projeto}
\end{figure}

%% ----------------------------------------------------------
\pagestyle{scrplain} 

%% ELEMENTOS POS-TEXTUAIS
\pagestyle{scrplain} 

\postextual

\bibliographystyle{abnt-alf}
\bibliography{bibliografia}

\end{document}
