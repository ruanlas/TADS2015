\chapter{CONCLUSÃO E TRABALHOS FUTUROS}\label{CAP5}

Os resultados comparativos obtidos foram bastante satisfatórios, o que pode ser comprovado em um primeiro momento pelo alto coeficiente de correlação obtido e a precisão dos resultados pode ser considerada elevada. Os erros relativos médios foram de aproximadamente de 1,02\% no caso de 3 leituras e de 2,43\% para 6 leituras, sendo razoavelmente baixos. Ressalta-se também que estes valores são próximos dos encontrados na literatura, o que é favorável à validação do método. Contudo, uma conclusão definitiva sobre a precisão não pode ser realizada, devido a limitações no estudo. A dificuldade em se obter amostras de pavimentos rígidos, por exemplo, impossibilitou uma padronização do ensaio o que causou diversas inconsistências na análise. 

Dentre as principais vantagens do método proposto destaca-se a facilidade de execução e a possibilidade de adaptação do procedimento, é possível por exemplo, realizar o escaneamento em campo evitando a extração de amostras. No caso de ser necessário extrair ou moldar corpos de prova, há também flexibilidade de tamanhos e formatos possíveis, uma vez que tanto a configuração cilíndrica quanto a placa foram adequadas. Contudo, para situações que exijam classificações quantitativas mais específicas, este procedimento mostrou-se insuficiente, uma vez que o parâmetro retornado refere-se somente à macrotextura, não sendo possível analisar os demais níveis de rugosidade. 

De uma forma geral conclui-se que os resultados pretendidos neste trabalho foram alcançados. Apesar de os resultados ainda serem insuficientes, o método analisado mostra-se promissor. Dessa forma, recomenda-se que seja definida uma padronização do procedimento e que mais medições venham a ser realizados a fim de confirmar a tendência observada. Sugere-se também a extensão de aplicação deste método a outros tipos de pavimento, como pavimentos mistos ou que façam uso de materiais alternativos. 
